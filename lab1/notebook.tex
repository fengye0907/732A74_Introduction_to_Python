
% Default to the notebook output style

    


% Inherit from the specified cell style.




    
\documentclass[11pt]{article}

    
    
    \usepackage[T1]{fontenc}
    % Nicer default font (+ math font) than Computer Modern for most use cases
    \usepackage{mathpazo}

    % Basic figure setup, for now with no caption control since it's done
    % automatically by Pandoc (which extracts ![](path) syntax from Markdown).
    \usepackage{graphicx}
    % We will generate all images so they have a width \maxwidth. This means
    % that they will get their normal width if they fit onto the page, but
    % are scaled down if they would overflow the margins.
    \makeatletter
    \def\maxwidth{\ifdim\Gin@nat@width>\linewidth\linewidth
    \else\Gin@nat@width\fi}
    \makeatother
    \let\Oldincludegraphics\includegraphics
    % Set max figure width to be 80% of text width, for now hardcoded.
    \renewcommand{\includegraphics}[1]{\Oldincludegraphics[width=.8\maxwidth]{#1}}
    % Ensure that by default, figures have no caption (until we provide a
    % proper Figure object with a Caption API and a way to capture that
    % in the conversion process - todo).
    \usepackage{caption}
    \DeclareCaptionLabelFormat{nolabel}{}
    \captionsetup{labelformat=nolabel}

    \usepackage{adjustbox} % Used to constrain images to a maximum size 
    \usepackage{xcolor} % Allow colors to be defined
    \usepackage{enumerate} % Needed for markdown enumerations to work
    \usepackage{geometry} % Used to adjust the document margins
    \usepackage{amsmath} % Equations
    \usepackage{amssymb} % Equations
    \usepackage{textcomp} % defines textquotesingle
    % Hack from http://tex.stackexchange.com/a/47451/13684:
    \AtBeginDocument{%
        \def\PYZsq{\textquotesingle}% Upright quotes in Pygmentized code
    }
    \usepackage{upquote} % Upright quotes for verbatim code
    \usepackage{eurosym} % defines \euro
    \usepackage[mathletters]{ucs} % Extended unicode (utf-8) support
    \usepackage[utf8x]{inputenc} % Allow utf-8 characters in the tex document
    \usepackage{fancyvrb} % verbatim replacement that allows latex
    \usepackage{grffile} % extends the file name processing of package graphics 
                         % to support a larger range 
    % The hyperref package gives us a pdf with properly built
    % internal navigation ('pdf bookmarks' for the table of contents,
    % internal cross-reference links, web links for URLs, etc.)
    \usepackage{hyperref}
    \usepackage{longtable} % longtable support required by pandoc >1.10
    \usepackage{booktabs}  % table support for pandoc > 1.12.2
    \usepackage[inline]{enumitem} % IRkernel/repr support (it uses the enumerate* environment)
    \usepackage[normalem]{ulem} % ulem is needed to support strikethroughs (\sout)
                                % normalem makes italics be italics, not underlines
    

    
    
    % Colors for the hyperref package
    \definecolor{urlcolor}{rgb}{0,.145,.698}
    \definecolor{linkcolor}{rgb}{.71,0.21,0.01}
    \definecolor{citecolor}{rgb}{.12,.54,.11}

    % ANSI colors
    \definecolor{ansi-black}{HTML}{3E424D}
    \definecolor{ansi-black-intense}{HTML}{282C36}
    \definecolor{ansi-red}{HTML}{E75C58}
    \definecolor{ansi-red-intense}{HTML}{B22B31}
    \definecolor{ansi-green}{HTML}{00A250}
    \definecolor{ansi-green-intense}{HTML}{007427}
    \definecolor{ansi-yellow}{HTML}{DDB62B}
    \definecolor{ansi-yellow-intense}{HTML}{B27D12}
    \definecolor{ansi-blue}{HTML}{208FFB}
    \definecolor{ansi-blue-intense}{HTML}{0065CA}
    \definecolor{ansi-magenta}{HTML}{D160C4}
    \definecolor{ansi-magenta-intense}{HTML}{A03196}
    \definecolor{ansi-cyan}{HTML}{60C6C8}
    \definecolor{ansi-cyan-intense}{HTML}{258F8F}
    \definecolor{ansi-white}{HTML}{C5C1B4}
    \definecolor{ansi-white-intense}{HTML}{A1A6B2}

    % commands and environments needed by pandoc snippets
    % extracted from the output of `pandoc -s`
    \providecommand{\tightlist}{%
      \setlength{\itemsep}{0pt}\setlength{\parskip}{0pt}}
    \DefineVerbatimEnvironment{Highlighting}{Verbatim}{commandchars=\\\{\}}
    % Add ',fontsize=\small' for more characters per line
    \newenvironment{Shaded}{}{}
    \newcommand{\KeywordTok}[1]{\textcolor[rgb]{0.00,0.44,0.13}{\textbf{{#1}}}}
    \newcommand{\DataTypeTok}[1]{\textcolor[rgb]{0.56,0.13,0.00}{{#1}}}
    \newcommand{\DecValTok}[1]{\textcolor[rgb]{0.25,0.63,0.44}{{#1}}}
    \newcommand{\BaseNTok}[1]{\textcolor[rgb]{0.25,0.63,0.44}{{#1}}}
    \newcommand{\FloatTok}[1]{\textcolor[rgb]{0.25,0.63,0.44}{{#1}}}
    \newcommand{\CharTok}[1]{\textcolor[rgb]{0.25,0.44,0.63}{{#1}}}
    \newcommand{\StringTok}[1]{\textcolor[rgb]{0.25,0.44,0.63}{{#1}}}
    \newcommand{\CommentTok}[1]{\textcolor[rgb]{0.38,0.63,0.69}{\textit{{#1}}}}
    \newcommand{\OtherTok}[1]{\textcolor[rgb]{0.00,0.44,0.13}{{#1}}}
    \newcommand{\AlertTok}[1]{\textcolor[rgb]{1.00,0.00,0.00}{\textbf{{#1}}}}
    \newcommand{\FunctionTok}[1]{\textcolor[rgb]{0.02,0.16,0.49}{{#1}}}
    \newcommand{\RegionMarkerTok}[1]{{#1}}
    \newcommand{\ErrorTok}[1]{\textcolor[rgb]{1.00,0.00,0.00}{\textbf{{#1}}}}
    \newcommand{\NormalTok}[1]{{#1}}
    
    % Additional commands for more recent versions of Pandoc
    \newcommand{\ConstantTok}[1]{\textcolor[rgb]{0.53,0.00,0.00}{{#1}}}
    \newcommand{\SpecialCharTok}[1]{\textcolor[rgb]{0.25,0.44,0.63}{{#1}}}
    \newcommand{\VerbatimStringTok}[1]{\textcolor[rgb]{0.25,0.44,0.63}{{#1}}}
    \newcommand{\SpecialStringTok}[1]{\textcolor[rgb]{0.73,0.40,0.53}{{#1}}}
    \newcommand{\ImportTok}[1]{{#1}}
    \newcommand{\DocumentationTok}[1]{\textcolor[rgb]{0.73,0.13,0.13}{\textit{{#1}}}}
    \newcommand{\AnnotationTok}[1]{\textcolor[rgb]{0.38,0.63,0.69}{\textbf{\textit{{#1}}}}}
    \newcommand{\CommentVarTok}[1]{\textcolor[rgb]{0.38,0.63,0.69}{\textbf{\textit{{#1}}}}}
    \newcommand{\VariableTok}[1]{\textcolor[rgb]{0.10,0.09,0.49}{{#1}}}
    \newcommand{\ControlFlowTok}[1]{\textcolor[rgb]{0.00,0.44,0.13}{\textbf{{#1}}}}
    \newcommand{\OperatorTok}[1]{\textcolor[rgb]{0.40,0.40,0.40}{{#1}}}
    \newcommand{\BuiltInTok}[1]{{#1}}
    \newcommand{\ExtensionTok}[1]{{#1}}
    \newcommand{\PreprocessorTok}[1]{\textcolor[rgb]{0.74,0.48,0.00}{{#1}}}
    \newcommand{\AttributeTok}[1]{\textcolor[rgb]{0.49,0.56,0.16}{{#1}}}
    \newcommand{\InformationTok}[1]{\textcolor[rgb]{0.38,0.63,0.69}{\textbf{\textit{{#1}}}}}
    \newcommand{\WarningTok}[1]{\textcolor[rgb]{0.38,0.63,0.69}{\textbf{\textit{{#1}}}}}
    
    
    % Define a nice break command that doesn't care if a line doesn't already
    % exist.
    \def\br{\hspace*{\fill} \\* }
    % Math Jax compatability definitions
    \def\gt{>}
    \def\lt{<}
    % Document parameters
    \title{732a74-la1-2019(1)}
    
    
    

    % Pygments definitions
    
\makeatletter
\def\PY@reset{\let\PY@it=\relax \let\PY@bf=\relax%
    \let\PY@ul=\relax \let\PY@tc=\relax%
    \let\PY@bc=\relax \let\PY@ff=\relax}
\def\PY@tok#1{\csname PY@tok@#1\endcsname}
\def\PY@toks#1+{\ifx\relax#1\empty\else%
    \PY@tok{#1}\expandafter\PY@toks\fi}
\def\PY@do#1{\PY@bc{\PY@tc{\PY@ul{%
    \PY@it{\PY@bf{\PY@ff{#1}}}}}}}
\def\PY#1#2{\PY@reset\PY@toks#1+\relax+\PY@do{#2}}

\expandafter\def\csname PY@tok@w\endcsname{\def\PY@tc##1{\textcolor[rgb]{0.73,0.73,0.73}{##1}}}
\expandafter\def\csname PY@tok@c\endcsname{\let\PY@it=\textit\def\PY@tc##1{\textcolor[rgb]{0.25,0.50,0.50}{##1}}}
\expandafter\def\csname PY@tok@cp\endcsname{\def\PY@tc##1{\textcolor[rgb]{0.74,0.48,0.00}{##1}}}
\expandafter\def\csname PY@tok@k\endcsname{\let\PY@bf=\textbf\def\PY@tc##1{\textcolor[rgb]{0.00,0.50,0.00}{##1}}}
\expandafter\def\csname PY@tok@kp\endcsname{\def\PY@tc##1{\textcolor[rgb]{0.00,0.50,0.00}{##1}}}
\expandafter\def\csname PY@tok@kt\endcsname{\def\PY@tc##1{\textcolor[rgb]{0.69,0.00,0.25}{##1}}}
\expandafter\def\csname PY@tok@o\endcsname{\def\PY@tc##1{\textcolor[rgb]{0.40,0.40,0.40}{##1}}}
\expandafter\def\csname PY@tok@ow\endcsname{\let\PY@bf=\textbf\def\PY@tc##1{\textcolor[rgb]{0.67,0.13,1.00}{##1}}}
\expandafter\def\csname PY@tok@nb\endcsname{\def\PY@tc##1{\textcolor[rgb]{0.00,0.50,0.00}{##1}}}
\expandafter\def\csname PY@tok@nf\endcsname{\def\PY@tc##1{\textcolor[rgb]{0.00,0.00,1.00}{##1}}}
\expandafter\def\csname PY@tok@nc\endcsname{\let\PY@bf=\textbf\def\PY@tc##1{\textcolor[rgb]{0.00,0.00,1.00}{##1}}}
\expandafter\def\csname PY@tok@nn\endcsname{\let\PY@bf=\textbf\def\PY@tc##1{\textcolor[rgb]{0.00,0.00,1.00}{##1}}}
\expandafter\def\csname PY@tok@ne\endcsname{\let\PY@bf=\textbf\def\PY@tc##1{\textcolor[rgb]{0.82,0.25,0.23}{##1}}}
\expandafter\def\csname PY@tok@nv\endcsname{\def\PY@tc##1{\textcolor[rgb]{0.10,0.09,0.49}{##1}}}
\expandafter\def\csname PY@tok@no\endcsname{\def\PY@tc##1{\textcolor[rgb]{0.53,0.00,0.00}{##1}}}
\expandafter\def\csname PY@tok@nl\endcsname{\def\PY@tc##1{\textcolor[rgb]{0.63,0.63,0.00}{##1}}}
\expandafter\def\csname PY@tok@ni\endcsname{\let\PY@bf=\textbf\def\PY@tc##1{\textcolor[rgb]{0.60,0.60,0.60}{##1}}}
\expandafter\def\csname PY@tok@na\endcsname{\def\PY@tc##1{\textcolor[rgb]{0.49,0.56,0.16}{##1}}}
\expandafter\def\csname PY@tok@nt\endcsname{\let\PY@bf=\textbf\def\PY@tc##1{\textcolor[rgb]{0.00,0.50,0.00}{##1}}}
\expandafter\def\csname PY@tok@nd\endcsname{\def\PY@tc##1{\textcolor[rgb]{0.67,0.13,1.00}{##1}}}
\expandafter\def\csname PY@tok@s\endcsname{\def\PY@tc##1{\textcolor[rgb]{0.73,0.13,0.13}{##1}}}
\expandafter\def\csname PY@tok@sd\endcsname{\let\PY@it=\textit\def\PY@tc##1{\textcolor[rgb]{0.73,0.13,0.13}{##1}}}
\expandafter\def\csname PY@tok@si\endcsname{\let\PY@bf=\textbf\def\PY@tc##1{\textcolor[rgb]{0.73,0.40,0.53}{##1}}}
\expandafter\def\csname PY@tok@se\endcsname{\let\PY@bf=\textbf\def\PY@tc##1{\textcolor[rgb]{0.73,0.40,0.13}{##1}}}
\expandafter\def\csname PY@tok@sr\endcsname{\def\PY@tc##1{\textcolor[rgb]{0.73,0.40,0.53}{##1}}}
\expandafter\def\csname PY@tok@ss\endcsname{\def\PY@tc##1{\textcolor[rgb]{0.10,0.09,0.49}{##1}}}
\expandafter\def\csname PY@tok@sx\endcsname{\def\PY@tc##1{\textcolor[rgb]{0.00,0.50,0.00}{##1}}}
\expandafter\def\csname PY@tok@m\endcsname{\def\PY@tc##1{\textcolor[rgb]{0.40,0.40,0.40}{##1}}}
\expandafter\def\csname PY@tok@gh\endcsname{\let\PY@bf=\textbf\def\PY@tc##1{\textcolor[rgb]{0.00,0.00,0.50}{##1}}}
\expandafter\def\csname PY@tok@gu\endcsname{\let\PY@bf=\textbf\def\PY@tc##1{\textcolor[rgb]{0.50,0.00,0.50}{##1}}}
\expandafter\def\csname PY@tok@gd\endcsname{\def\PY@tc##1{\textcolor[rgb]{0.63,0.00,0.00}{##1}}}
\expandafter\def\csname PY@tok@gi\endcsname{\def\PY@tc##1{\textcolor[rgb]{0.00,0.63,0.00}{##1}}}
\expandafter\def\csname PY@tok@gr\endcsname{\def\PY@tc##1{\textcolor[rgb]{1.00,0.00,0.00}{##1}}}
\expandafter\def\csname PY@tok@ge\endcsname{\let\PY@it=\textit}
\expandafter\def\csname PY@tok@gs\endcsname{\let\PY@bf=\textbf}
\expandafter\def\csname PY@tok@gp\endcsname{\let\PY@bf=\textbf\def\PY@tc##1{\textcolor[rgb]{0.00,0.00,0.50}{##1}}}
\expandafter\def\csname PY@tok@go\endcsname{\def\PY@tc##1{\textcolor[rgb]{0.53,0.53,0.53}{##1}}}
\expandafter\def\csname PY@tok@gt\endcsname{\def\PY@tc##1{\textcolor[rgb]{0.00,0.27,0.87}{##1}}}
\expandafter\def\csname PY@tok@err\endcsname{\def\PY@bc##1{\setlength{\fboxsep}{0pt}\fcolorbox[rgb]{1.00,0.00,0.00}{1,1,1}{\strut ##1}}}
\expandafter\def\csname PY@tok@kc\endcsname{\let\PY@bf=\textbf\def\PY@tc##1{\textcolor[rgb]{0.00,0.50,0.00}{##1}}}
\expandafter\def\csname PY@tok@kd\endcsname{\let\PY@bf=\textbf\def\PY@tc##1{\textcolor[rgb]{0.00,0.50,0.00}{##1}}}
\expandafter\def\csname PY@tok@kn\endcsname{\let\PY@bf=\textbf\def\PY@tc##1{\textcolor[rgb]{0.00,0.50,0.00}{##1}}}
\expandafter\def\csname PY@tok@kr\endcsname{\let\PY@bf=\textbf\def\PY@tc##1{\textcolor[rgb]{0.00,0.50,0.00}{##1}}}
\expandafter\def\csname PY@tok@bp\endcsname{\def\PY@tc##1{\textcolor[rgb]{0.00,0.50,0.00}{##1}}}
\expandafter\def\csname PY@tok@fm\endcsname{\def\PY@tc##1{\textcolor[rgb]{0.00,0.00,1.00}{##1}}}
\expandafter\def\csname PY@tok@vc\endcsname{\def\PY@tc##1{\textcolor[rgb]{0.10,0.09,0.49}{##1}}}
\expandafter\def\csname PY@tok@vg\endcsname{\def\PY@tc##1{\textcolor[rgb]{0.10,0.09,0.49}{##1}}}
\expandafter\def\csname PY@tok@vi\endcsname{\def\PY@tc##1{\textcolor[rgb]{0.10,0.09,0.49}{##1}}}
\expandafter\def\csname PY@tok@vm\endcsname{\def\PY@tc##1{\textcolor[rgb]{0.10,0.09,0.49}{##1}}}
\expandafter\def\csname PY@tok@sa\endcsname{\def\PY@tc##1{\textcolor[rgb]{0.73,0.13,0.13}{##1}}}
\expandafter\def\csname PY@tok@sb\endcsname{\def\PY@tc##1{\textcolor[rgb]{0.73,0.13,0.13}{##1}}}
\expandafter\def\csname PY@tok@sc\endcsname{\def\PY@tc##1{\textcolor[rgb]{0.73,0.13,0.13}{##1}}}
\expandafter\def\csname PY@tok@dl\endcsname{\def\PY@tc##1{\textcolor[rgb]{0.73,0.13,0.13}{##1}}}
\expandafter\def\csname PY@tok@s2\endcsname{\def\PY@tc##1{\textcolor[rgb]{0.73,0.13,0.13}{##1}}}
\expandafter\def\csname PY@tok@sh\endcsname{\def\PY@tc##1{\textcolor[rgb]{0.73,0.13,0.13}{##1}}}
\expandafter\def\csname PY@tok@s1\endcsname{\def\PY@tc##1{\textcolor[rgb]{0.73,0.13,0.13}{##1}}}
\expandafter\def\csname PY@tok@mb\endcsname{\def\PY@tc##1{\textcolor[rgb]{0.40,0.40,0.40}{##1}}}
\expandafter\def\csname PY@tok@mf\endcsname{\def\PY@tc##1{\textcolor[rgb]{0.40,0.40,0.40}{##1}}}
\expandafter\def\csname PY@tok@mh\endcsname{\def\PY@tc##1{\textcolor[rgb]{0.40,0.40,0.40}{##1}}}
\expandafter\def\csname PY@tok@mi\endcsname{\def\PY@tc##1{\textcolor[rgb]{0.40,0.40,0.40}{##1}}}
\expandafter\def\csname PY@tok@il\endcsname{\def\PY@tc##1{\textcolor[rgb]{0.40,0.40,0.40}{##1}}}
\expandafter\def\csname PY@tok@mo\endcsname{\def\PY@tc##1{\textcolor[rgb]{0.40,0.40,0.40}{##1}}}
\expandafter\def\csname PY@tok@ch\endcsname{\let\PY@it=\textit\def\PY@tc##1{\textcolor[rgb]{0.25,0.50,0.50}{##1}}}
\expandafter\def\csname PY@tok@cm\endcsname{\let\PY@it=\textit\def\PY@tc##1{\textcolor[rgb]{0.25,0.50,0.50}{##1}}}
\expandafter\def\csname PY@tok@cpf\endcsname{\let\PY@it=\textit\def\PY@tc##1{\textcolor[rgb]{0.25,0.50,0.50}{##1}}}
\expandafter\def\csname PY@tok@c1\endcsname{\let\PY@it=\textit\def\PY@tc##1{\textcolor[rgb]{0.25,0.50,0.50}{##1}}}
\expandafter\def\csname PY@tok@cs\endcsname{\let\PY@it=\textit\def\PY@tc##1{\textcolor[rgb]{0.25,0.50,0.50}{##1}}}

\def\PYZbs{\char`\\}
\def\PYZus{\char`\_}
\def\PYZob{\char`\{}
\def\PYZcb{\char`\}}
\def\PYZca{\char`\^}
\def\PYZam{\char`\&}
\def\PYZlt{\char`\<}
\def\PYZgt{\char`\>}
\def\PYZsh{\char`\#}
\def\PYZpc{\char`\%}
\def\PYZdl{\char`\$}
\def\PYZhy{\char`\-}
\def\PYZsq{\char`\'}
\def\PYZdq{\char`\"}
\def\PYZti{\char`\~}
% for compatibility with earlier versions
\def\PYZat{@}
\def\PYZlb{[}
\def\PYZrb{]}
\makeatother


    % Exact colors from NB
    \definecolor{incolor}{rgb}{0.0, 0.0, 0.5}
    \definecolor{outcolor}{rgb}{0.545, 0.0, 0.0}



    
    % Prevent overflowing lines due to hard-to-break entities
    \sloppy 
    % Setup hyperref package
    \hypersetup{
      breaklinks=true,  % so long urls are correctly broken across lines
      colorlinks=true,
      urlcolor=urlcolor,
      linkcolor=linkcolor,
      citecolor=citecolor,
      }
    % Slightly bigger margins than the latex defaults
    
    \geometry{verbose,tmargin=1in,bmargin=1in,lmargin=1in,rmargin=1in}
    
    

    \begin{document}
    
    
    \maketitle
    
    

    
    \section{Lab 1: Python basics}\label{lab-1-python-basics}

    \textbf{Student I:} andch552 (Andreas Christopoulos Charitos)

\textbf{Student II:} zijfe244 (Zijie Feng)

    \subsubsection{A word of caution}\label{a-word-of-caution}

There are currently two versions of Python in common use, Python 2 and
Python 3, which are not 100\% compatible. Python 2 is slowly being
phased out but has a large enough install base to still be relevant.
This course uses the more modern Python 3 but while searching for help
online it is not uncommon to find help for Python 2. Especially older
posts on sources such as Stack Exchange might refer to Python 2 as
simply "Python". This should not cause any serious problems but keep it
in mind whenever googling. With regards to this lab, the largest
differences are how \texttt{print} works and the best practice
recommendations for string formatting.

    \subsubsection{References to R}\label{references-to-r}

Most students taking this course who are not already familiar with
Python will probably have some experience of the R programming language.
For this reason, there will be intermittent references to R throughout
this lab. For those of you with a background in R (or MATLAB/Octave, or
Julia) the most important thing to remember is that indexing starts at
0, not at 1.

    \subsubsection{Recommended Reading}\label{recommended-reading}

This course is not built on any specific source and no specific
litterature is required. However, for those who prefer to have a printed
reference book, we recommended the books by Mark Lutz:

\begin{itemize}
\item
  Learning Python by Mark Lutz, 5th edition, O'Reilly. Recommended for
  those who have no experience of Python. This book is called LP in the
  text below.
\item
  Programming Python by Mark Lutz, 4th edition, O'Reilly. Recommended
  for those who have some experience with Python, it generally covers
  more advanced topics than what is included in this course but gives
  you a chance to dig a bit deeper if you're already comfortable with
  the basics. This book is called PP in the text.
\end{itemize}

For the student interested in Python as a language, it is worth
mentioning * Fluent Python by Luciano Ramalho (also O'Reilly). Note that
it is - at the time of writing - still in its first edition, from 2015.
Thus newer features will be missing.

    \subsubsection{A note about notebooks}\label{a-note-about-notebooks}

When using this notebook, you can enter python code in the empty cells,
then press ctrl-enter. The code in the cell is executed and if any
output occurs it will be displayed below the square. Code executed in
this manner will use the same environment regardless of where in the
notebook document it is placed. This means that variables and functions
assigned values in one cell will thereafter be accessible from all other
cells in your notebook session.

Note that the programming environments described in section 1 of LP is
not applicable when you run python in this notebook.

    \subsubsection{A note about the structure of this
lab}\label{a-note-about-the-structure-of-this-lab}

This lab will contain tasks of varying difficulty. There might be cases
when the solution seems too simple to be true (in retrospect), and cases
where you have seen similar material elsewhere in the course. Don't be
fooled by this. In many cases, the task might just serve to remind us of
things that are worthwhile to check out, or to find out how to use a
specific method.

We will be returning to, and using, several of the concepts in this lab.

    \subsubsection{1. Strings and string
handling}\label{strings-and-string-handling}

The primary datatype for storing raw text in Python is the string. Note
that there is no character datatype, only strings of length 1. This can
be compared to how there are no atomic numbers in R, only vectors of
length 1. A reference to the string datatype can be found
\textbf{\href{https://docs.python.org/3/library/stdtypes.html\#text-sequence-type-str}{here}}.

{[}Litterature: LP: Part II, especially Chapter 4, 7.{]}

    \begin{enumerate}
\def\labelenumi{\alph{enumi})}
\tightlist
\item
  Define the variable \texttt{parrot} as the string containing the
  sentence \emph{It is dead, that is what is wrong with it. This is an
  ex-"Parrot".}.
\end{enumerate}

{[}Note: If you have been programming in a language such as C or Java,
you might be a bit confused about the term "define". Different languages
use different terms when creating variables, such as "define",
"declare", "initialize", etc. with slightly different meanings. In
statically typed languages such as C or Java, declaring a variable
creates a name connected to a container which can contain data of a
specific type, but does not put a value in that container.
Initialization is then the act of putting an initial value in such a
container. Defining a variable is often used as a synonym to declaring a
variable in statically typed languages but as Python is dynamically
typed, i.e. variables can contain values of any type, there is no need
to declare variables before initializing them. Thus, defining a variable
in python entails simply assigning a value to a new name, at which point
the variable is both declared and initialized. This works exactly as in
R.{]}

    \begin{Verbatim}[commandchars=\\\{\}]
{\color{incolor}In [{\color{incolor}8}]:} \PY{n}{parrot}\PY{o}{=}\PY{l+s+s1}{\PYZsq{}}\PY{l+s+s1}{\PYZdq{}}\PY{l+s+s1}{It is dead, that is what is wrong with it. This is an ex\PYZhy{}}\PY{l+s+s1}{\PYZdq{}}\PY{l+s+s1}{Parrot}\PY{l+s+s1}{\PYZdq{}}\PY{l+s+s1}{\PYZsq{}}
        \PY{n+nb}{print}\PY{p}{(}\PY{n}{parrot}\PY{p}{)}
        \PY{n+nb}{type}\PY{p}{(}\PY{n}{parrot}\PY{p}{)}
\end{Verbatim}


    \begin{Verbatim}[commandchars=\\\{\}]
"It is dead, that is what is wrong with it. This is an ex-"Parrot"

    \end{Verbatim}

\begin{Verbatim}[commandchars=\\\{\}]
{\color{outcolor}Out[{\color{outcolor}8}]:} str
\end{Verbatim}
            
    \begin{enumerate}
\def\labelenumi{\alph{enumi})}
\setcounter{enumi}{1}
\tightlist
\item
  What methods does the string now called \texttt{parrot} (or indeed any
  string) seem to support? Write Python commands below to find out.
\end{enumerate}

    \begin{Verbatim}[commandchars=\\\{\}]
{\color{incolor}In [{\color{incolor}9}]:} \PY{n+nb}{dir}\PY{p}{(}\PY{n}{parrot}\PY{p}{)}
\end{Verbatim}


\begin{Verbatim}[commandchars=\\\{\}]
{\color{outcolor}Out[{\color{outcolor}9}]:} ['\_\_add\_\_',
         '\_\_class\_\_',
         '\_\_contains\_\_',
         '\_\_delattr\_\_',
         '\_\_dir\_\_',
         '\_\_doc\_\_',
         '\_\_eq\_\_',
         '\_\_format\_\_',
         '\_\_ge\_\_',
         '\_\_getattribute\_\_',
         '\_\_getitem\_\_',
         '\_\_getnewargs\_\_',
         '\_\_gt\_\_',
         '\_\_hash\_\_',
         '\_\_init\_\_',
         '\_\_init\_subclass\_\_',
         '\_\_iter\_\_',
         '\_\_le\_\_',
         '\_\_len\_\_',
         '\_\_lt\_\_',
         '\_\_mod\_\_',
         '\_\_mul\_\_',
         '\_\_ne\_\_',
         '\_\_new\_\_',
         '\_\_reduce\_\_',
         '\_\_reduce\_ex\_\_',
         '\_\_repr\_\_',
         '\_\_rmod\_\_',
         '\_\_rmul\_\_',
         '\_\_setattr\_\_',
         '\_\_sizeof\_\_',
         '\_\_str\_\_',
         '\_\_subclasshook\_\_',
         'capitalize',
         'casefold',
         'center',
         'count',
         'encode',
         'endswith',
         'expandtabs',
         'find',
         'format',
         'format\_map',
         'index',
         'isalnum',
         'isalpha',
         'isdecimal',
         'isdigit',
         'isidentifier',
         'islower',
         'isnumeric',
         'isprintable',
         'isspace',
         'istitle',
         'isupper',
         'join',
         'ljust',
         'lower',
         'lstrip',
         'maketrans',
         'partition',
         'replace',
         'rfind',
         'rindex',
         'rjust',
         'rpartition',
         'rsplit',
         'rstrip',
         'split',
         'splitlines',
         'startswith',
         'strip',
         'swapcase',
         'title',
         'translate',
         'upper',
         'zfill']
\end{Verbatim}
            
    \begin{enumerate}
\def\labelenumi{\alph{enumi})}
\setcounter{enumi}{2}
\tightlist
\item
  Count the number of characters (letters, blank space, commas, periods
  etc) in the sentence.
\end{enumerate}

    \begin{Verbatim}[commandchars=\\\{\}]
{\color{incolor}In [{\color{incolor}10}]:} \PY{n+nb}{print}\PY{p}{(}\PY{l+s+s2}{\PYZdq{}}\PY{l+s+s2}{The number of characters in parrot string are : }\PY{l+s+si}{\PYZob{}0\PYZcb{}}\PY{l+s+s2}{\PYZdq{}} \PY{o}{.}\PY{n}{format}\PY{p}{(}\PY{n+nb}{len}\PY{p}{(}\PY{n}{parrot}\PY{p}{)}\PY{p}{)}\PY{p}{)}
\end{Verbatim}


    \begin{Verbatim}[commandchars=\\\{\}]
The number of characters in parrot string are : 66

    \end{Verbatim}

    \begin{enumerate}
\def\labelenumi{\alph{enumi})}
\setcounter{enumi}{4}
\tightlist
\item
  If we type \texttt{parrot\ +\ parrot}, should it change the string
  itself, or merely produce a new string? How would you test your
  intuition? Write expressions below.
\end{enumerate}

    \begin{Verbatim}[commandchars=\\\{\}]
{\color{incolor}In [{\color{incolor}11}]:} \PY{c+c1}{\PYZsh{} it produce a new string}
         \PY{n+nb}{print}\PY{p}{(}\PY{n+nb}{type}\PY{p}{(}\PY{n}{parrot}\PY{o}{+}\PY{n}{parrot}\PY{p}{)}\PY{p}{)}
         \PY{n+nb}{print}\PY{p}{(}\PY{n}{parrot}\PY{p}{)}
\end{Verbatim}


    \begin{Verbatim}[commandchars=\\\{\}]
<class 'str'>
"It is dead, that is what is wrong with it. This is an ex-"Parrot"

    \end{Verbatim}

    \begin{enumerate}
\def\labelenumi{\alph{enumi})}
\setcounter{enumi}{5}
\tightlist
\item
  Separate the sentence into a list of words (possibly including
  separators) using a built-in method. Call the list
  \texttt{parrot\_words}.
\end{enumerate}

    \begin{Verbatim}[commandchars=\\\{\}]
{\color{incolor}In [{\color{incolor}12}]:} \PY{n}{parrot\PYZus{}words}\PY{o}{=}\PY{n}{parrot}\PY{o}{.}\PY{n}{split}\PY{p}{(}\PY{p}{)}
         \PY{n}{parrot\PYZus{}words}
\end{Verbatim}


\begin{Verbatim}[commandchars=\\\{\}]
{\color{outcolor}Out[{\color{outcolor}12}]:} ['"It',
          'is',
          'dead,',
          'that',
          'is',
          'what',
          'is',
          'wrong',
          'with',
          'it.',
          'This',
          'is',
          'an',
          'ex-"Parrot"']
\end{Verbatim}
            
    \begin{enumerate}
\def\labelenumi{\alph{enumi})}
\setcounter{enumi}{4}
\tightlist
\item
  Merge (concatenate) \texttt{parrot\_words} into a string again.
\end{enumerate}

    \begin{Verbatim}[commandchars=\\\{\}]
{\color{incolor}In [{\color{incolor}13}]:} \PY{n}{merged\PYZus{}parrot}\PY{o}{=}\PY{l+s+s2}{\PYZdq{}}\PY{l+s+s2}{ }\PY{l+s+s2}{\PYZdq{}}\PY{o}{.}\PY{n}{join}\PY{p}{(}\PY{n}{parrot\PYZus{}words}\PY{p}{)}
         \PY{n}{merged\PYZus{}parrot}
\end{Verbatim}


\begin{Verbatim}[commandchars=\\\{\}]
{\color{outcolor}Out[{\color{outcolor}13}]:} '"It is dead, that is what is wrong with it. This is an ex-"Parrot"'
\end{Verbatim}
            
    \subsubsection{2. Iteration, sequences and string
formatting}\label{iteration-sequences-and-string-formatting}

Loops are not as painfully slow in Python as they are in R and thus, not
as critical to avoid. However, for many use cases,
\emph{comprehensions}, like \emph{list comprehensions} or \emph{dict
comprehensions} are faster. In this assignment we will see both
traditional loop constructs and comprehensions. For an introduction to
comprehensions,
\textbf{\href{https://python-3-patterns-idioms-test.readthedocs.io/en/latest/Comprehensions.html}{this}}
might be a good place to start.

It should also be noted that what Python calls lists are unnamed
sequences. As in R, a Python list can contain elements of many types,
however, these can only be accessed by indexing or sequence, not by name
as in R.

    \begin{enumerate}
\def\labelenumi{\alph{enumi})}
\tightlist
\item
  Write a \texttt{for}-loop that produces the following output on the
  screen: \textgreater{}
  \texttt{The\ next\ number\ in\ the\ loop\ is\ 5} \textgreater{}
  \texttt{The\ next\ number\ in\ the\ loop\ is\ 6} \textgreater{} ...
  \textgreater{} \texttt{The\ next\ number\ in\ the\ loop\ is\ 10}
\end{enumerate}

{[}Hint: the \texttt{range} function has more than one argument.{]}
{[}Literature: For the range construct see LP part II chapter 4
(p.112).{]}

    \begin{Verbatim}[commandchars=\\\{\}]
{\color{incolor}In [{\color{incolor}14}]:} \PY{k}{for} \PY{n}{i} \PY{o+ow}{in} \PY{n+nb}{range}\PY{p}{(}\PY{l+m+mi}{5}\PY{p}{,}\PY{l+m+mi}{11}\PY{p}{,}\PY{l+m+mi}{1}\PY{p}{)}\PY{p}{:}
             \PY{n+nb}{print}\PY{p}{(}\PY{l+s+s2}{\PYZdq{}}\PY{l+s+s2}{The next number in the loop is }\PY{l+s+si}{\PYZob{}0\PYZcb{}}\PY{l+s+s2}{\PYZdq{}}\PY{o}{.}\PY{n}{format}\PY{p}{(}\PY{n}{i}\PY{p}{)}\PY{p}{)}
\end{Verbatim}


    \begin{Verbatim}[commandchars=\\\{\}]
The next number in the loop is 5
The next number in the loop is 6
The next number in the loop is 7
The next number in the loop is 8
The next number in the loop is 9
The next number in the loop is 10

    \end{Verbatim}

    \begin{enumerate}
\def\labelenumi{\alph{enumi})}
\setcounter{enumi}{1}
\tightlist
\item
  Write a \texttt{for}-loop that for a given\texttt{n} sets
  \texttt{first\_n\_squared} to the sum of squares of the first
  \texttt{n} numbers (0..n-1).
\end{enumerate}

    \begin{Verbatim}[commandchars=\\\{\}]
{\color{incolor}In [{\color{incolor}15}]:} \PY{n}{n} \PY{o}{=} \PY{l+m+mi}{100}  \PY{c+c1}{\PYZsh{} If we change this and run the code, the value of first\PYZus{}n\PYZus{}squared should change afterwards!}
         \PY{c+c1}{\PYZsh{} your code goes here}
         \PY{n}{first\PYZus{}n\PYZus{}squared}\PY{o}{=}\PY{l+m+mi}{0}
         \PY{k}{for} \PY{n}{i} \PY{o+ow}{in} \PY{n+nb}{range}\PY{p}{(}\PY{l+m+mi}{0}\PY{p}{,}\PY{n}{n}\PY{p}{)}\PY{p}{:}
             \PY{n}{first\PYZus{}n\PYZus{}squared}\PY{o}{+}\PY{o}{=}\PY{n}{i}\PY{o}{*}\PY{o}{*}\PY{l+m+mi}{2}
         
         \PY{n}{first\PYZus{}n\PYZus{}squared}   \PY{c+c1}{\PYZsh{} should return 0\PYZca{}2 + 1\PYZca{}2 + ... + 99\PYZca{}2 = 328350 if n = 100}
\end{Verbatim}


\begin{Verbatim}[commandchars=\\\{\}]
{\color{outcolor}Out[{\color{outcolor}15}]:} 328350
\end{Verbatim}
            
    \begin{enumerate}
\def\labelenumi{\alph{enumi})}
\setcounter{enumi}{2}
\tightlist
\item
  Write a code snippet that counts the number of \textbf{letters} in
  \texttt{parrot} (as defined above). Use a \texttt{for} loop.
\end{enumerate}

    \begin{Verbatim}[commandchars=\\\{\}]
{\color{incolor}In [{\color{incolor}16}]:} \PY{n}{letters}\PY{o}{=}\PY{l+m+mi}{0}
         \PY{k}{for}  \PY{n}{i} \PY{o+ow}{in} \PY{n+nb}{range}\PY{p}{(}\PY{n+nb}{len}\PY{p}{(}\PY{n}{parrot}\PY{p}{)}\PY{p}{)}\PY{p}{:}
             \PY{k}{if} \PY{n}{parrot}\PY{p}{[}\PY{n}{i}\PY{p}{]}\PY{o}{.}\PY{n}{isalpha}\PY{p}{(}\PY{p}{)}\PY{p}{:}
                 \PY{n}{letters}\PY{o}{+}\PY{o}{=}\PY{l+m+mi}{1}
             \PY{k}{else}\PY{p}{:}
                 \PY{k}{continue}
                 
         \PY{n+nb}{print}\PY{p}{(}\PY{l+s+s2}{\PYZdq{}}\PY{l+s+s2}{The number of letters in parrot is: }\PY{l+s+si}{\PYZob{}0\PYZcb{}}\PY{l+s+s2}{\PYZdq{}}\PY{o}{.}\PY{n}{format}\PY{p}{(}\PY{n}{letters}\PY{p}{)}\PY{p}{)} 
\end{Verbatim}


    \begin{Verbatim}[commandchars=\\\{\}]
The number of letters in parrot is: 47

    \end{Verbatim}

    \begin{Verbatim}[commandchars=\\\{\}]
{\color{incolor}In [{\color{incolor}17}]:} \PY{n}{names} \PY{o}{=} \PY{p}{[}\PY{l+s+s1}{\PYZsq{}}\PY{l+s+s1}{Tesco}\PY{l+s+s1}{\PYZsq{}}\PY{p}{,} \PY{l+s+s1}{\PYZsq{}}\PY{l+s+s1}{Forex}\PY{l+s+s1}{\PYZsq{}}\PY{p}{,} \PY{l+s+s1}{\PYZsq{}}\PY{l+s+s1}{Alonzo}\PY{l+s+s1}{\PYZsq{}}\PY{p}{,} \PY{l+s+s1}{\PYZsq{}}\PY{l+s+s1}{Zeno}\PY{l+s+s1}{\PYZsq{}}\PY{p}{]}
\end{Verbatim}


    \begin{enumerate}
\def\labelenumi{\alph{enumi})}
\setcounter{enumi}{3}
\tightlist
\item
  Write a for-loop that iterates over the list \texttt{names} below and
  presents them on the screen in the following fashion:
\end{enumerate}

\begin{quote}
\texttt{The\ name\ Tesco\ is\ nice} ...
\texttt{The\ name\ Zeno\ is\ nice}
\end{quote}

Use Python's string formatting capabilities (the \texttt{format}
function in the string class) to solve the problem.

{[}Warning: The best practices for how to do string formatting differs
from Python 2 and 3, make sure you use the Python 3 approach.{]}
{[}Literature: String formatting is covered in LP part II chapter 7.{]}

    \begin{Verbatim}[commandchars=\\\{\}]
{\color{incolor}In [{\color{incolor}18}]:} \PY{k}{for} \PY{n}{name} \PY{o+ow}{in} \PY{n}{names}\PY{p}{:}
             \PY{n+nb}{print} \PY{p}{(}\PY{l+s+s2}{\PYZdq{}}\PY{l+s+s2}{The name }\PY{l+s+si}{\PYZob{}0\PYZcb{}}\PY{l+s+s2}{ is nice}\PY{l+s+s2}{\PYZdq{}}\PY{o}{.}\PY{n}{format}\PY{p}{(}\PY{n}{name}\PY{p}{)}\PY{p}{)}
\end{Verbatim}


    \begin{Verbatim}[commandchars=\\\{\}]
The name Tesco is nice
The name Forex is nice
The name Alonzo is nice
The name Zeno is nice

    \end{Verbatim}

    \begin{enumerate}
\def\labelenumi{\alph{enumi})}
\setcounter{enumi}{4}
\tightlist
\item
  Write a for-loop that iterates over the list \texttt{names} and
  produces the list \texttt{n\_letters} (\texttt{{[}5,5,6,4{]}}) with
  the length of each name.
\end{enumerate}

    \begin{Verbatim}[commandchars=\\\{\}]
{\color{incolor}In [{\color{incolor}19}]:} \PY{n}{n\PYZus{}letters}\PY{o}{=}\PY{n+nb}{list}\PY{p}{(}\PY{p}{)}
         \PY{k}{for} \PY{n}{name} \PY{o+ow}{in} \PY{n}{names}\PY{p}{:}
             \PY{n}{n\PYZus{}letters}\PY{o}{.}\PY{n}{append}\PY{p}{(}\PY{n+nb}{len}\PY{p}{(}\PY{n}{name}\PY{p}{)}\PY{p}{)}
         
         \PY{n}{n\PYZus{}letters}    
\end{Verbatim}


\begin{Verbatim}[commandchars=\\\{\}]
{\color{outcolor}Out[{\color{outcolor}19}]:} [5, 5, 6, 4]
\end{Verbatim}
            
    \begin{enumerate}
\def\labelenumi{\alph{enumi})}
\setcounter{enumi}{5}
\tightlist
\item
  How would you - in a Python interpreter/REPL or in this Notebook -
  retrieve the help for the built-in function \texttt{max}?
\end{enumerate}

    \begin{Verbatim}[commandchars=\\\{\}]
{\color{incolor}In [{\color{incolor}20}]:} \PY{n}{help}\PY{p}{(}\PY{n+nb}{max}\PY{p}{)}
\end{Verbatim}


    \begin{Verbatim}[commandchars=\\\{\}]
Help on built-in function max in module builtins:

max({\ldots})
    max(iterable, *[, default=obj, key=func]) -> value
    max(arg1, arg2, *args, *[, key=func]) -> value
    
    With a single iterable argument, return its biggest item. The
    default keyword-only argument specifies an object to return if
    the provided iterable is empty.
    With two or more arguments, return the largest argument.


    \end{Verbatim}

    \begin{enumerate}
\def\labelenumi{\alph{enumi})}
\setcounter{enumi}{6}
\tightlist
\item
  Show an example of how \texttt{max} can be used with an iterable of
  your choice.
\end{enumerate}

    \begin{Verbatim}[commandchars=\\\{\}]
{\color{incolor}In [{\color{incolor}21}]:} \PY{n}{str0}\PY{o}{=}\PY{l+s+s2}{\PYZdq{}}\PY{l+s+s2}{My salsa makes all the pretty girls to drink and dance oooo, my salsa, oooooo}\PY{l+s+s2}{\PYZdq{}}
         \PY{n}{str1}\PY{o}{=}\PY{n}{str0}\PY{o}{.}\PY{n}{replace}\PY{p}{(}\PY{l+s+s2}{\PYZdq{}}\PY{l+s+s2}{,}\PY{l+s+s2}{\PYZdq{}}\PY{p}{,}\PY{l+s+s2}{\PYZdq{}}\PY{l+s+s2}{\PYZdq{}}\PY{p}{)}
         
         \PY{n}{l}\PY{o}{=}\PY{n+nb}{list}\PY{p}{(}\PY{p}{)}
         \PY{k}{for} \PY{n}{word} \PY{o+ow}{in} \PY{n}{str1}\PY{o}{.}\PY{n}{split}\PY{p}{(}\PY{p}{)}\PY{p}{:}
             \PY{n}{l}\PY{o}{.}\PY{n}{append}\PY{p}{(}\PY{n+nb}{len}\PY{p}{(}\PY{n}{word}\PY{p}{)}\PY{p}{)}
         \PY{n}{idx}\PY{o}{=}\PY{n}{l}\PY{o}{.}\PY{n}{index}\PY{p}{(}\PY{n+nb}{max}\PY{p}{(}\PY{n}{l}\PY{p}{)}\PY{p}{)}    
         \PY{n+nb}{print}\PY{p}{(}\PY{l+s+s2}{\PYZdq{}}\PY{l+s+s2}{The first word with most letters is : }\PY{l+s+se}{\PYZbs{}\PYZsq{}}\PY{l+s+si}{\PYZob{}0\PYZcb{}}\PY{l+s+se}{\PYZbs{}\PYZsq{}}\PY{l+s+s2}{ with }\PY{l+s+si}{\PYZob{}1\PYZcb{}}\PY{l+s+s2}{ letters }\PY{l+s+s2}{\PYZdq{}}\PY{o}{.}\PY{n}{format}\PY{p}{(}\PY{n}{str1}\PY{o}{.}\PY{n}{split}\PY{p}{(}\PY{p}{)}\PY{p}{[}\PY{n}{idx}\PY{p}{]}\PY{p}{,} \PY{n+nb}{max}\PY{p}{(}\PY{n}{l}\PY{p}{)}\PY{p}{)}\PY{p}{)}
\end{Verbatim}


    \begin{Verbatim}[commandchars=\\\{\}]
The first word with most letters is : 'pretty' with 6 letters 

    \end{Verbatim}

    \begin{enumerate}
\def\labelenumi{\alph{enumi})}
\setcounter{enumi}{7}
\tightlist
\item
  Use a comprehension (or generator) to calculate the sum 0\^{}2 + ... +
  (n-1)\^{}2 as above.
\end{enumerate}

    \begin{Verbatim}[commandchars=\\\{\}]
{\color{incolor}In [{\color{incolor}22}]:} \PY{n}{n} \PY{o}{=} \PY{l+m+mi}{100}
         \PY{n}{first\PYZus{}n\PYZus{}squared} \PY{o}{=}\PY{n+nb}{sum}\PY{p}{(}\PY{p}{[} \PY{n}{x}\PY{o}{*}\PY{o}{*}\PY{l+m+mi}{2} \PY{k}{for} \PY{n}{x} \PY{o+ow}{in} \PY{n+nb}{range}\PY{p}{(}\PY{l+m+mi}{100}\PY{p}{)}\PY{p}{]}\PY{p}{)} \PY{c+c1}{\PYZsh{} Change None to your solution.}
         \PY{n}{first\PYZus{}n\PYZus{}squared} \PY{c+c1}{\PYZsh{} Should return the same result as your for\PYZhy{}loop.}
\end{Verbatim}


\begin{Verbatim}[commandchars=\\\{\}]
{\color{outcolor}Out[{\color{outcolor}22}]:} 328350
\end{Verbatim}
            
    \begin{enumerate}
\def\labelenumi{\roman{enumi})}
\tightlist
\item
  Solve assignment e) using a list comprehension.
\end{enumerate}

{[}Literature: Comprehensions are covered in LP part II chapter 4.{]}

    \begin{Verbatim}[commandchars=\\\{\}]
{\color{incolor}In [{\color{incolor}23}]:} \PY{n}{n\PYZus{}letters} \PY{o}{=} \PY{p}{[}\PY{n+nb}{len}\PY{p}{(}\PY{n}{name}\PY{p}{)} \PY{k}{for} \PY{n}{name} \PY{o+ow}{in} \PY{n}{names} \PY{k}{if} \PY{n}{name}\PY{p}{]}
         
         \PY{n}{n\PYZus{}letters}
\end{Verbatim}


\begin{Verbatim}[commandchars=\\\{\}]
{\color{outcolor}Out[{\color{outcolor}23}]:} [5, 5, 6, 4]
\end{Verbatim}
            
    \begin{enumerate}
\def\labelenumi{\alph{enumi})}
\setcounter{enumi}{9}
\tightlist
\item
  Use a list comprehension to produce a list \texttt{short\_long} that
  indicates if the name (in the list \texttt{names}) has more than four
  letters. The answer should be
  \texttt{{[}\textquotesingle{}long\textquotesingle{},\ \textquotesingle{}long\textquotesingle{},\ \textquotesingle{}long\textquotesingle{},\ \textquotesingle{}short\textquotesingle{}{]}}.
\end{enumerate}

    \begin{Verbatim}[commandchars=\\\{\}]
{\color{incolor}In [{\color{incolor}24}]:} \PY{n}{short\PYZus{}long}\PY{o}{=}\PY{p}{[}\PY{l+s+s2}{\PYZdq{}}\PY{l+s+s2}{long}\PY{l+s+s2}{\PYZdq{}} \PY{k}{if} \PY{n+nb}{len}\PY{p}{(}\PY{n}{name}\PY{p}{)}\PY{o}{\PYZgt{}}\PY{l+m+mi}{4} \PY{k}{else} \PY{l+s+s2}{\PYZdq{}}\PY{l+s+s2}{short}\PY{l+s+s2}{\PYZdq{}} \PY{k}{for} \PY{n}{name} \PY{o+ow}{in} \PY{n}{names}\PY{p}{]}
         
         \PY{n}{short\PYZus{}long}
\end{Verbatim}


\begin{Verbatim}[commandchars=\\\{\}]
{\color{outcolor}Out[{\color{outcolor}24}]:} ['long', 'long', 'long', 'short']
\end{Verbatim}
            
    \begin{enumerate}
\def\labelenumi{\alph{enumi})}
\setcounter{enumi}{10}
\tightlist
\item
  Use a comprehension to count the number of letters in \texttt{parrot}.
  You may not use a \texttt{for}-loop. (The comprehension will contain
  the word \texttt{for}, but it isn't a
  \texttt{for\ ...\ in\ ...:}-statement.)
\end{enumerate}

    \begin{Verbatim}[commandchars=\\\{\}]
{\color{incolor}In [{\color{incolor}25}]:} \PY{n+nb}{sum}\PY{p}{(}\PY{p}{[}\PY{n}{parrot}\PY{p}{[}\PY{n}{char}\PY{p}{]}\PY{o}{.}\PY{n}{isalpha}\PY{p}{(}\PY{p}{)} \PY{k}{for} \PY{n}{char} \PY{o+ow}{in} \PY{n+nb}{range}\PY{p}{(}\PY{n+nb}{len}\PY{p}{(}\PY{n}{parrot}\PY{p}{)}\PY{p}{)}\PY{p}{]}\PY{p}{)}
\end{Verbatim}


\begin{Verbatim}[commandchars=\\\{\}]
{\color{outcolor}Out[{\color{outcolor}25}]:} 47
\end{Verbatim}
            
    {[}Note: this is fairly similar to the long/short task, but note how we
access member functions of the values.{]}

    \begin{enumerate}
\def\labelenumi{\alph{enumi})}
\setcounter{enumi}{11}
\tightlist
\item
  Below we have the string \texttt{datadump}. Retrieve the substring
  string starting at character 27 and ending at character 34 by means of
  slicing.
\end{enumerate}

    \begin{Verbatim}[commandchars=\\\{\}]
{\color{incolor}In [{\color{incolor}26}]:} \PY{n}{datadump} \PY{o}{=} \PY{l+s+s2}{\PYZdq{}}\PY{l+s+s2}{The name of the game is \PYZlt{}b\PYZgt{}old html\PYZlt{}/b\PYZgt{}. That is \PYZlt{}b\PYZgt{}so cool\PYZlt{}/b\PYZgt{}.}\PY{l+s+s2}{\PYZdq{}}
\end{Verbatim}


    \begin{enumerate}
\def\labelenumi{\alph{enumi})}
\setcounter{enumi}{11}
\tightlist
\item
  Write a loop that uses indices to \textbf{simultaneously} loop over
  the lists \texttt{names} and \texttt{short\_long} to write the
  following to the screen:
\end{enumerate}

\begin{quote}
\texttt{The\ name\ Tesco\ is\ a\ long\ name} ...
\texttt{The\ name\ Zeno\ is\ a\ short\ name}
\end{quote}

    \begin{Verbatim}[commandchars=\\\{\}]
{\color{incolor}In [{\color{incolor}27}]:} \PY{k}{for} \PY{n}{i} \PY{o+ow}{in} \PY{n+nb}{range}\PY{p}{(}\PY{n+nb}{len}\PY{p}{(}\PY{n}{names}\PY{p}{)}\PY{p}{)}\PY{p}{:}
             \PY{n+nb}{print}\PY{p}{(}\PY{l+s+s2}{\PYZdq{}}\PY{l+s+s2}{The name }\PY{l+s+si}{\PYZob{}0\PYZcb{}}\PY{l+s+s2}{ is a }\PY{l+s+si}{\PYZob{}1\PYZcb{}}\PY{l+s+s2}{ name}\PY{l+s+s2}{\PYZdq{}}\PY{o}{.}\PY{n}{format}\PY{p}{(}\PY{n}{names}\PY{p}{[}\PY{n}{i}\PY{p}{]}\PY{p}{,}\PY{n}{short\PYZus{}long}\PY{p}{[}\PY{n}{i}\PY{p}{]}\PY{p}{)}\PY{p}{)}
\end{Verbatim}


    \begin{Verbatim}[commandchars=\\\{\}]
The name Tesco is a long name
The name Forex is a long name
The name Alonzo is a long name
The name Zeno is a short name

    \end{Verbatim}

    Note: this is a common programming pattern, though not particularly
Pythonic in this use case. We do however need to know how to use indices
in lists to work properly with Python.

    \begin{enumerate}
\def\labelenumi{\alph{enumi})}
\setcounter{enumi}{12}
\tightlist
\item
  Do the task above once more, but this time without the use of indices.
\end{enumerate}

    \begin{Verbatim}[commandchars=\\\{\}]
{\color{incolor}In [{\color{incolor}28}]:} \PY{k}{for} \PY{n}{name}\PY{p}{,}\PY{n}{short} \PY{o+ow}{in} \PY{n+nb}{zip}\PY{p}{(}\PY{n}{names}\PY{p}{,}\PY{n}{short\PYZus{}long}\PY{p}{)}\PY{p}{:}
             \PY{n+nb}{print}\PY{p}{(}\PY{l+s+s2}{\PYZdq{}}\PY{l+s+s2}{The name }\PY{l+s+si}{\PYZob{}0\PYZcb{}}\PY{l+s+s2}{ is a }\PY{l+s+si}{\PYZob{}1\PYZcb{}}\PY{l+s+s2}{ name}\PY{l+s+s2}{\PYZdq{}}\PY{o}{.}\PY{n}{format}\PY{p}{(}\PY{n}{name}\PY{p}{,}\PY{n}{short}\PY{p}{)}\PY{p}{)}
\end{Verbatim}


    \begin{Verbatim}[commandchars=\\\{\}]
The name Tesco is a long name
The name Forex is a long name
The name Alonzo is a long name
The name Zeno is a short name

    \end{Verbatim}

    {[}Hint: Use the \texttt{zip} function.{]} {[}Literature: zip usage with
dictionary is found in LP part II chapter 8 and dictionary
comprehensions in the same place.{]}

    \begin{enumerate}
\def\labelenumi{\alph{enumi})}
\setcounter{enumi}{13}
\tightlist
\item
  Among the built-in datatypes, it is also worth mentioning the tuple.
  Construct two tuples, \texttt{one} containing the number one and
  \texttt{two} containing the number 1 and the number 2. What happens if
  you add them? Name some method that a list with similar content (such
  as \texttt{two\_list} below) would support, that \texttt{two} doesn't
  and explain why this makes sense.
\end{enumerate}

    \begin{Verbatim}[commandchars=\\\{\}]
{\color{incolor}In [{\color{incolor}29}]:} \PY{n}{one} \PY{o}{=} \PY{p}{(}\PY{l+m+mi}{1}\PY{p}{,}\PY{p}{)}    \PY{c+c1}{\PYZsh{} Change this.}
         \PY{n}{two} \PY{o}{=} \PY{p}{(}\PY{l+m+mi}{1}\PY{p}{,}\PY{l+m+mi}{2}\PY{p}{)}    \PY{c+c1}{\PYZsh{} Change this}
         \PY{n}{two\PYZus{}list} \PY{o}{=} \PY{p}{[}\PY{l+m+mi}{1}\PY{p}{,} \PY{l+m+mi}{2}\PY{p}{]}
         
         \PY{n+nb}{print}\PY{p}{(}\PY{n}{one}\PY{o}{+}\PY{n}{two}\PY{p}{)}
         \PY{n+nb}{print}\PY{p}{(}\PY{n+nb}{set}\PY{p}{(}\PY{n}{one}\PY{o}{+}\PY{n}{two}\PY{p}{)}\PY{p}{)}
         \PY{n+nb}{print}\PY{p}{(}\PY{n+nb}{list}\PY{p}{(}\PY{n}{two}\PY{p}{)}\PY{p}{)}
         
         \PY{n+nb}{print}\PY{p}{(}\PY{l+s+s2}{\PYZdq{}\PYZdq{}\PYZdq{}}
         \PY{l+s+s2}{Special methods of }\PY{l+s+s2}{\PYZdq{}}\PY{l+s+s2}{list}\PY{l+s+s2}{\PYZdq{}}\PY{l+s+s2}{:}
         \PY{l+s+s2}{append()}
         \PY{l+s+s2}{pop()}
         \PY{l+s+s2}{remover()}
         \PY{l+s+s2}{insert()}
         \PY{l+s+s2}{...}
         
         \PY{l+s+s2}{Since }\PY{l+s+s2}{\PYZdq{}}\PY{l+s+s2}{tuple}\PY{l+s+s2}{\PYZdq{}}\PY{l+s+s2}{ is a datatype does not have attribute }\PY{l+s+s2}{\PYZdq{}}\PY{l+s+s2}{\PYZus{}\PYZus{}init\PYZus{}\PYZus{}()}\PY{l+s+s2}{\PYZdq{}}\PY{l+s+s2}{, it cannot change the elements inside.}
         \PY{l+s+s2}{But }\PY{l+s+s2}{\PYZdq{}}\PY{l+s+s2}{list}\PY{l+s+s2}{\PYZdq{}}\PY{l+s+s2}{ is a datatype which can change its own elements.}
         \PY{l+s+s2}{\PYZdq{}\PYZdq{}\PYZdq{}}\PY{p}{)}
\end{Verbatim}


    \begin{Verbatim}[commandchars=\\\{\}]
(1, 1, 2)
\{1, 2\}
[1, 2]

Special methods of "list":
append()
pop()
remover()
insert()
{\ldots}

Since "tuple" is a datatype does not have attribute "\_\_init\_\_()", it cannot change the elements inside.
But "list" is a datatype which can change its own elements.


    \end{Verbatim}

    \subsubsection{3. Conditionals, logic and while
loops}\label{conditionals-logic-and-while-loops}

    \begin{enumerate}
\def\labelenumi{\alph{enumi})}
\tightlist
\item
  Below we have an integer called \texttt{n}. Write code that prints
  "It's even!" if it is even, and "It's odd!" if it's not.
\end{enumerate}

    \begin{Verbatim}[commandchars=\\\{\}]
{\color{incolor}In [{\color{incolor}30}]:} \PY{n}{n} \PY{o}{=} \PY{l+m+mi}{256} \PY{c+c1}{\PYZsh{} Change this to other values and run your code to test.}
         \PY{c+c1}{\PYZsh{} Your code here.}
         \PY{k}{def} \PY{n+nf}{eo}\PY{p}{(}\PY{n}{n}\PY{p}{)}\PY{p}{:}
             \PY{k}{if} \PY{n}{n}\PY{o}{\PYZpc{}}\PY{k}{2}==0:
                 \PY{n+nb}{print}\PY{p}{(}\PY{l+s+s2}{\PYZdq{}}\PY{l+s+s2}{It}\PY{l+s+s2}{\PYZsq{}}\PY{l+s+s2}{s even!}\PY{l+s+s2}{\PYZdq{}}\PY{p}{)}
             \PY{k}{else} \PY{p}{:}
                 \PY{n+nb}{print}\PY{p}{(}\PY{l+s+s2}{\PYZdq{}}\PY{l+s+s2}{It}\PY{l+s+s2}{\PYZsq{}}\PY{l+s+s2}{s odd!}\PY{l+s+s2}{\PYZdq{}}\PY{p}{)}
         \PY{n}{eo}\PY{p}{(}\PY{n}{n}\PY{p}{)}
         \PY{n}{eo}\PY{p}{(}\PY{l+m+mi}{3}\PY{p}{)}
\end{Verbatim}


    \begin{Verbatim}[commandchars=\\\{\}]
It's even!
It's odd!

    \end{Verbatim}

    \begin{enumerate}
\def\labelenumi{\alph{enumi})}
\setcounter{enumi}{1}
\tightlist
\item
  Below we have the list \texttt{options}. Write code (including an
  \texttt{if} statement) that ensures that the boolean variable
  \texttt{OPTIMIZE} is True \emph{if and only if} the list contains the
  string \texttt{-\/-optimize} (exactly like that).
\end{enumerate}

    \begin{Verbatim}[commandchars=\\\{\}]
{\color{incolor}In [{\color{incolor}31}]:} \PY{n}{OPTIMIZE} \PY{o}{=} \PY{k+kc}{None}       \PY{c+c1}{\PYZsh{} Or some value which we are unsure of.}
         \PY{n}{options} \PY{o}{=} \PY{p}{[}\PY{l+s+s1}{\PYZsq{}}\PY{l+s+s1}{\PYZhy{}\PYZhy{}print\PYZhy{}results}\PY{l+s+s1}{\PYZsq{}}\PY{p}{,} \PY{l+s+s1}{\PYZsq{}}\PY{l+s+s1}{\PYZhy{}\PYZhy{}optimize}\PY{l+s+s1}{\PYZsq{}}\PY{p}{,} \PY{l+s+s1}{\PYZsq{}}\PY{l+s+s1}{\PYZhy{}x}\PY{l+s+s1}{\PYZsq{}}\PY{p}{]}  \PY{c+c1}{\PYZsh{} This might have been generated by a GUI or command line option}
         
         \PY{c+c1}{\PYZsh{} Your code goes here.}
         \PY{k}{if} \PY{l+s+s2}{\PYZdq{}}\PY{l+s+s2}{\PYZhy{}\PYZhy{}optimize}\PY{l+s+s2}{\PYZdq{}} \PY{o+ow}{in} \PY{n}{options}\PY{p}{:}
             \PY{n}{OPTIMIZE}\PY{o}{=}\PY{k+kc}{True}
         
         \PY{c+c1}{\PYZsh{} Here OPTIMIZE should be True if and only if we found \PYZsq{}\PYZhy{}\PYZhy{}optimize\PYZsq{} in the list.}
         \PY{n}{OPTIMIZE}
\end{Verbatim}


\begin{Verbatim}[commandchars=\\\{\}]
{\color{outcolor}Out[{\color{outcolor}31}]:} True
\end{Verbatim}
            
    Note: It might be tempting to use a \texttt{for} loop. In this case, we
will not be needing this, and you may \emph{not} use it. Python has some
useful built-ins to test for membership.

You may use an \texttt{else}-free \texttt{if} statement if you like.

    \begin{enumerate}
\def\labelenumi{\alph{enumi})}
\setcounter{enumi}{2}
\tightlist
\item
  Redo the task above, but now consider the case where the boolean
  \texttt{OPTIMIZE} is True \emph{if and only if} the \texttt{options}
  list contains either \texttt{-\/-optimize} or \texttt{-o} (or both).
\end{enumerate}

    \begin{Verbatim}[commandchars=\\\{\}]
{\color{incolor}In [{\color{incolor}32}]:} \PY{n}{OPTIMIZE} \PY{o}{=} \PY{k+kc}{None}       \PY{c+c1}{\PYZsh{} Or some value which we are unsure of.}
         \PY{n}{options} \PY{o}{=} \PY{p}{[}\PY{l+s+s1}{\PYZsq{}}\PY{l+s+s1}{\PYZhy{}\PYZhy{}print\PYZhy{}results}\PY{l+s+s1}{\PYZsq{}}\PY{p}{,} \PY{l+s+s1}{\PYZsq{}}\PY{l+s+s1}{\PYZhy{}o}\PY{l+s+s1}{\PYZsq{}}\PY{p}{,} \PY{l+s+s1}{\PYZsq{}}\PY{l+s+s1}{\PYZhy{}x}\PY{l+s+s1}{\PYZsq{}}\PY{p}{]}  \PY{c+c1}{\PYZsh{} This might have been generated by a GUI or command line option}
         
         \PY{c+c1}{\PYZsh{} Your code goes here.}
         \PY{k}{def} \PY{n+nf}{opt}\PY{p}{(}\PY{n}{options}\PY{p}{)}\PY{p}{:}
             \PY{k}{if} \PY{p}{(}\PY{l+s+s2}{\PYZdq{}}\PY{l+s+s2}{\PYZhy{}\PYZhy{}optimize}\PY{l+s+s2}{\PYZdq{}} \PY{o+ow}{in} \PY{n}{options}\PY{p}{)} \PY{o}{|} \PY{p}{(}\PY{l+s+s2}{\PYZdq{}}\PY{l+s+s2}{\PYZhy{}o}\PY{l+s+s2}{\PYZdq{}} \PY{o+ow}{in} \PY{n}{options}\PY{p}{)}\PY{p}{:}
                 \PY{n}{OPTIMIZE}\PY{o}{=}\PY{k+kc}{True}
                 \PY{n+nb}{print}\PY{p}{(}\PY{n}{OPTIMIZE}\PY{p}{)}
             \PY{k}{else}\PY{p}{:}
                 \PY{n}{OPTIMIZE}\PY{o}{=}\PY{k+kc}{None}
                 \PY{n+nb}{print}\PY{p}{(}\PY{n}{OPTIMIZE}\PY{p}{)}
         
         \PY{c+c1}{\PYZsh{} Here OPTIMIZE should be True if and only if we found \PYZsq{}\PYZhy{}\PYZhy{}optimize\PYZsq{} or \PYZsq{}\PYZhy{}o\PYZsq{} in the list.}
         \PY{n}{opt}\PY{p}{(}\PY{n}{options}\PY{p}{)}
         \PY{n}{opt}\PY{p}{(}\PY{p}{[}\PY{l+s+s1}{\PYZsq{}}\PY{l+s+s1}{\PYZhy{}\PYZhy{}optimize}\PY{l+s+s1}{\PYZsq{}}\PY{p}{]}\PY{p}{)}
         \PY{n}{opt}\PY{p}{(}\PY{p}{[}\PY{l+s+s1}{\PYZsq{}}\PY{l+s+s1}{\PYZhy{}o}\PY{l+s+s1}{\PYZsq{}}\PY{p}{,}\PY{l+s+s1}{\PYZsq{}}\PY{l+s+s1}{\PYZhy{}\PYZhy{}optimize}\PY{l+s+s1}{\PYZsq{}}\PY{p}{]}\PY{p}{)}
\end{Verbatim}


    \begin{Verbatim}[commandchars=\\\{\}]
True
True
True

    \end{Verbatim}

    {[}Hint: Don't forget to test your code with different versions of the
options list!

If you find something that seems strange, you might want to check what
the value of the \emph{condition itself} is.{]}

{[}Note: This extension of the task is included as it includes a common
source of hard-to-spot bugs.{]}

    \begin{enumerate}
\def\labelenumi{\alph{enumi})}
\setcounter{enumi}{3}
\tightlist
\item
  Write out \emph{a few} good tests that should (in an \emph{extremely}
  informal sense) illustrate the correctness of your solution. Make sure
  that you actually try them out with your code.
\end{enumerate}

Also make sure to write what the cases actually tell you, and why they
are useful inputs to your set of tests. If you already have the test
\texttt{options\ =\ {[}"hey"{]}}, adding the test
\texttt{options\ =\ {[}"hello"{]}} doesn't add any possible ways of
failing (or any coverage).

    \begin{Verbatim}[commandchars=\\\{\}]
{\color{incolor}In [{\color{incolor}33}]:} \PY{c+c1}{\PYZsh{} Test 0: if options is the list below, OPTIMIZE should be True after. }
         \PY{n}{options} \PY{o}{=} \PY{p}{[}\PY{l+s+s1}{\PYZsq{}}\PY{l+s+s1}{\PYZhy{}o}\PY{l+s+s1}{\PYZsq{}}\PY{p}{,}\PY{l+s+s1}{\PYZsq{}}\PY{l+s+s1}{hello}\PY{l+s+s1}{\PYZsq{}}\PY{p}{,}\PY{l+m+mi}{1}\PY{p}{]}
         \PY{n}{opt}\PY{p}{(}\PY{p}{[}\PY{l+s+s1}{\PYZsq{}}\PY{l+s+s1}{hello}\PY{l+s+s1}{\PYZsq{}}\PY{p}{]}\PY{p}{)}    \PY{c+c1}{\PYZsh{} None}
         \PY{n}{opt}\PY{p}{(}\PY{p}{[}\PY{l+m+mi}{1}\PY{p}{]}\PY{p}{)}        \PY{c+c1}{\PYZsh{} None}
         \PY{n}{opt}\PY{p}{(}\PY{n}{options}\PY{p}{)}
         
         \PY{c+c1}{\PYZsh{} This test demonstrates that ...}
         \PY{n+nb}{print}\PY{p}{(}\PY{l+s+s2}{\PYZdq{}\PYZdq{}\PYZdq{}}
         \PY{l+s+s2}{We try }\PY{l+s+s2}{\PYZdq{}}\PY{l+s+s2}{str}\PY{l+s+s2}{\PYZdq{}}\PY{l+s+s2}{ and }\PY{l+s+s2}{\PYZdq{}}\PY{l+s+s2}{integer}\PY{l+s+s2}{\PYZdq{}}\PY{l+s+s2}{ as input, the solutions should be }\PY{l+s+s2}{\PYZdq{}}\PY{l+s+s2}{None}\PY{l+s+s2}{\PYZdq{}}\PY{l+s+s2}{ and the }
         \PY{l+s+s2}{results confirm. However, if the list }\PY{l+s+s2}{\PYZdq{}}\PY{l+s+s2}{options}\PY{l+s+s2}{\PYZdq{}}\PY{l+s+s2}{ contains }\PY{l+s+s2}{\PYZdq{}}\PY{l+s+s2}{\PYZhy{}o}\PY{l+s+s2}{\PYZdq{}}\PY{l+s+s2}{, the OPTIMIZE }
         \PY{l+s+s2}{would be }\PY{l+s+s2}{\PYZdq{}}\PY{l+s+s2}{True}\PY{l+s+s2}{\PYZdq{}}\PY{l+s+s2}{.}
         \PY{l+s+s2}{\PYZdq{}\PYZdq{}\PYZdq{}}\PY{p}{)}
\end{Verbatim}


    \begin{Verbatim}[commandchars=\\\{\}]
None
None
True

We try "str" and "integer" as input, the solutions should be "None" and the 
results confirm. However, if the list "options" contains "-o", the OPTIMIZE 
would be "True".


    \end{Verbatim}

    {[}Note: This way of testing your code is very primitive, but it's good
to get used to constructing test cases. We will be discussing procedures
and functions in the next lab.{]}

    \begin{enumerate}
\def\labelenumi{\alph{enumi})}
\setcounter{enumi}{3}
\tightlist
\item
  Sometimes we can avoid using an \texttt{if} statement altogether. The
  task above is a prime example of this (and was introduced to get some
  practice with the \texttt{if} statement). Solve the task above in a
  one-liner without resorting to an \texttt{if} statement. (You may use
  an \texttt{if} expression, but you don't have to.)
\end{enumerate}

    \begin{Verbatim}[commandchars=\\\{\}]
{\color{incolor}In [{\color{incolor}34}]:} \PY{n}{options} \PY{o}{=} \PY{p}{[}\PY{l+s+s1}{\PYZsq{}}\PY{l+s+s1}{\PYZhy{}\PYZhy{}print\PYZhy{}results}\PY{l+s+s1}{\PYZsq{}}\PY{p}{,} \PY{l+s+s1}{\PYZsq{}}\PY{l+s+s1}{\PYZhy{}o}\PY{l+s+s1}{\PYZsq{}}\PY{p}{,} \PY{l+s+s1}{\PYZsq{}}\PY{l+s+s1}{\PYZhy{}x}\PY{l+s+s1}{\PYZsq{}}\PY{p}{]}  \PY{c+c1}{\PYZsh{} This might have been generated by a GUI or command line option}
         
         \PY{n}{OPTIMIZE}\PY{o}{=}\PY{p}{(}\PY{l+s+s1}{\PYZsq{}}\PY{l+s+s1}{\PYZhy{}o}\PY{l+s+s1}{\PYZsq{}} \PY{o+ow}{in} \PY{n}{options}\PY{p}{)} \PY{o+ow}{or} \PY{p}{(}\PY{l+s+s1}{\PYZsq{}}\PY{l+s+s1}{\PYZhy{}optimize}\PY{l+s+s1}{\PYZsq{}} \PY{o+ow}{in} \PY{n}{options}\PY{p}{)}
         
         \PY{c+c1}{\PYZsh{} Here OPTIMIZE should be True if and only if we found \PYZsq{}\PYZhy{}\PYZhy{}optimize\PYZsq{} or \PYZsq{}\PYZhy{}o\PYZsq{} in the list.}
         
         \PY{n}{OPTIMIZE}
\end{Verbatim}


\begin{Verbatim}[commandchars=\\\{\}]
{\color{outcolor}Out[{\color{outcolor}34}]:} True
\end{Verbatim}
            
    {[}Hint: What should the value of the condition be when you enter the
then-branch of the \texttt{if}? When you enter the else-branch?{]}

    \begin{enumerate}
\def\labelenumi{\alph{enumi})}
\setcounter{enumi}{4}
\tightlist
\item
  Write a \texttt{while}-loop that repeatedly generates a random number
  from a uniform distribution over the interval {[}0,1{]}, and prints
  the sentence 'The random number is smaller than 0.9' on the screen
  until the generated random number is greater than 0.9.
\end{enumerate}

{[}Hint: Python has a \texttt{random} module with basic random number
generators.{]}

{[}Literature: Introduction to the Random module can be found in LP part
III chapter 5 (Numeric Types). Importing modules is introduced in part I
chapter 3 and covered in depth in part IV.{]}

    \begin{Verbatim}[commandchars=\\\{\}]
{\color{incolor}In [{\color{incolor}35}]:} \PY{k+kn}{import} \PY{n+nn}{random}
         \PY{n}{number}\PY{o}{=}\PY{l+m+mf}{0.89}
         \PY{n}{random}\PY{o}{.}\PY{n}{seed}\PY{p}{(}\PY{l+m+mi}{123456}\PY{p}{)}
         
         \PY{k}{while} \PY{n}{number}\PY{o}{\PYZlt{}}\PY{l+m+mf}{0.9}\PY{p}{:}
             \PY{n}{number}\PY{o}{=}\PY{n}{random}\PY{o}{.}\PY{n}{uniform}\PY{p}{(}\PY{l+m+mi}{0}\PY{p}{,}\PY{l+m+mi}{1}\PY{p}{)}
             \PY{k}{if} \PY{n}{number}\PY{o}{\PYZlt{}}\PY{l+m+mf}{0.9} \PY{p}{:}
                 \PY{n+nb}{print}\PY{p}{(}\PY{l+s+s2}{\PYZdq{}}\PY{l+s+s2}{The random number }\PY{l+s+si}{\PYZob{}0\PYZcb{}}\PY{l+s+s2}{ is smaller than 0.9}\PY{l+s+s2}{\PYZdq{}}\PY{o}{.}\PY{n}{format}\PY{p}{(}\PY{n}{number}\PY{p}{)}\PY{p}{)}
             \PY{k}{else} \PY{p}{:}
                 \PY{n+nb}{print} \PY{p}{(}\PY{l+s+s2}{\PYZdq{}}\PY{l+s+s2}{The random number }\PY{l+s+si}{\PYZob{}0\PYZcb{}}\PY{l+s+s2}{ is greater than 0.9 exiting while loop}\PY{l+s+s2}{\PYZdq{}}\PY{o}{.}\PY{n}{format}\PY{p}{(}\PY{n}{number}\PY{p}{)}\PY{p}{)}
\end{Verbatim}


    \begin{Verbatim}[commandchars=\\\{\}]
The random number 0.8056271362589 is smaller than 0.9
The random number 0.7940590105180981 is smaller than 0.9
The random number 0.029425761106168014 is smaller than 0.9
The random number 0.17465638335376021 is smaller than 0.9
The random number 0.0022298761599784944 is smaller than 0.9
The random number 0.6638830667081945 is smaller than 0.9
The random number 0.07704930045464609 is smaller than 0.9
The random number 0.26852339527315083 is smaller than 0.9
The random number 0.11639518985546105 is smaller than 0.9
The random number 0.2290251898688216 is smaller than 0.9
The random number 0.48741758541138713 is smaller than 0.9
The random number 0.18610490054852236 is smaller than 0.9
The random number 0.024271797606526646 is smaller than 0.9
The random number 0.9135736087631979 is greater than 0.9 exiting while loop

    \end{Verbatim}

    \subsubsection{4. Dictionaries}\label{dictionaries}

Dictionaries are association tables, or maps, connecting a key to a
value. For instance a name represented by a string as key with a number
representing some attribute as a value. Dictionaries can themselves be
values in other dictionaries, creating nested or hierarchical data
structures. This is similar to named lists in R but keys in Python
dictionaries can be more complex than just strings.

{[}Literature: Dictionaries are found in LP section II chapter 4.{]}

    \begin{enumerate}
\def\labelenumi{\alph{enumi})}
\tightlist
\item
  Make a dictionary named \texttt{amadeus} containing the information
  that the student Amadeus is a male, scored 8 on the Algebra exam and
  13 on the History exam. The dictionary should NOT include a name
  entry.
\end{enumerate}

    \begin{Verbatim}[commandchars=\\\{\}]
{\color{incolor}In [{\color{incolor}36}]:} \PY{n}{amadeus}\PY{o}{=}\PY{p}{\PYZob{}}\PY{l+s+s2}{\PYZdq{}}\PY{l+s+s2}{Name}\PY{l+s+s2}{\PYZdq{}}\PY{p}{:}\PY{l+s+s2}{\PYZdq{}}\PY{l+s+s2}{Amadeus}\PY{l+s+s2}{\PYZdq{}}\PY{p}{,}\PY{l+s+s2}{\PYZdq{}}\PY{l+s+s2}{Gender}\PY{l+s+s2}{\PYZdq{}}\PY{p}{:}\PY{l+s+s2}{\PYZdq{}}\PY{l+s+s2}{Male}\PY{l+s+s2}{\PYZdq{}}\PY{p}{,}\PY{l+s+s2}{\PYZdq{}}\PY{l+s+s2}{Algebra}\PY{l+s+s2}{\PYZdq{}}\PY{p}{:}\PY{l+m+mi}{8}\PY{p}{,}\PY{l+s+s2}{\PYZdq{}}\PY{l+s+s2}{History}\PY{l+s+s2}{\PYZdq{}}\PY{p}{:}\PY{l+m+mi}{13}\PY{p}{\PYZcb{}}
\end{Verbatim}


    \begin{enumerate}
\def\labelenumi{\alph{enumi})}
\setcounter{enumi}{1}
\tightlist
\item
  Make three more dictionaries, one for each of the students: Rosa, Mona
  and Ludwig, from the information in the following table:
\end{enumerate}

\begin{longtable}[]{@{}cccc@{}}
\toprule
Name & Gender & Algebra & History\tabularnewline
\midrule
\endhead
Rosa & Female & 19 & 22\tabularnewline
Mona & Female & 6 & 27\tabularnewline
Ludwig & Other & 12 & 18\tabularnewline
\bottomrule
\end{longtable}

    \begin{Verbatim}[commandchars=\\\{\}]
{\color{incolor}In [{\color{incolor}37}]:} \PY{n}{rosa}\PY{o}{=}\PY{p}{\PYZob{}}\PY{l+s+s2}{\PYZdq{}}\PY{l+s+s2}{Name}\PY{l+s+s2}{\PYZdq{}}\PY{p}{:}\PY{l+s+s2}{\PYZdq{}}\PY{l+s+s2}{Rosa}\PY{l+s+s2}{\PYZdq{}}\PY{p}{,}\PY{l+s+s2}{\PYZdq{}}\PY{l+s+s2}{Gender}\PY{l+s+s2}{\PYZdq{}}\PY{p}{:}\PY{l+s+s2}{\PYZdq{}}\PY{l+s+s2}{Female}\PY{l+s+s2}{\PYZdq{}}\PY{p}{,}\PY{l+s+s2}{\PYZdq{}}\PY{l+s+s2}{Algebra}\PY{l+s+s2}{\PYZdq{}}\PY{p}{:}\PY{l+m+mi}{19}\PY{p}{,}\PY{l+s+s2}{\PYZdq{}}\PY{l+s+s2}{History}\PY{l+s+s2}{\PYZdq{}}\PY{p}{:}\PY{l+m+mi}{22}\PY{p}{\PYZcb{}}
         \PY{n}{mona}\PY{o}{=}\PY{p}{\PYZob{}}\PY{l+s+s2}{\PYZdq{}}\PY{l+s+s2}{Name}\PY{l+s+s2}{\PYZdq{}}\PY{p}{:}\PY{l+s+s2}{\PYZdq{}}\PY{l+s+s2}{Mona}\PY{l+s+s2}{\PYZdq{}}\PY{p}{,}\PY{l+s+s2}{\PYZdq{}}\PY{l+s+s2}{Gender}\PY{l+s+s2}{\PYZdq{}}\PY{p}{:}\PY{l+s+s2}{\PYZdq{}}\PY{l+s+s2}{Female}\PY{l+s+s2}{\PYZdq{}}\PY{p}{,}\PY{l+s+s2}{\PYZdq{}}\PY{l+s+s2}{Algebra}\PY{l+s+s2}{\PYZdq{}}\PY{p}{:}\PY{l+m+mi}{6}\PY{p}{,}\PY{l+s+s2}{\PYZdq{}}\PY{l+s+s2}{History}\PY{l+s+s2}{\PYZdq{}}\PY{p}{:}\PY{l+m+mi}{27}\PY{p}{\PYZcb{}}
         \PY{n}{ludwig}\PY{o}{=}\PY{p}{\PYZob{}}\PY{l+s+s2}{\PYZdq{}}\PY{l+s+s2}{Name}\PY{l+s+s2}{\PYZdq{}}\PY{p}{:}\PY{l+s+s2}{\PYZdq{}}\PY{l+s+s2}{Ludwig}\PY{l+s+s2}{\PYZdq{}}\PY{p}{,}\PY{l+s+s2}{\PYZdq{}}\PY{l+s+s2}{Gender}\PY{l+s+s2}{\PYZdq{}}\PY{p}{:}\PY{l+s+s2}{\PYZdq{}}\PY{l+s+s2}{Other}\PY{l+s+s2}{\PYZdq{}}\PY{p}{,}\PY{l+s+s2}{\PYZdq{}}\PY{l+s+s2}{Algebra}\PY{l+s+s2}{\PYZdq{}}\PY{p}{:}\PY{l+m+mi}{12}\PY{p}{,}\PY{l+s+s2}{\PYZdq{}}\PY{l+s+s2}{History}\PY{l+s+s2}{\PYZdq{}}\PY{p}{:}\PY{l+m+mi}{18}\PY{p}{\PYZcb{}}
\end{Verbatim}


    \begin{enumerate}
\def\labelenumi{\alph{enumi})}
\setcounter{enumi}{2}
\tightlist
\item
  Combine the four students in a dictionary named \texttt{students} such
  that a user of your dictionary can type
  \texttt{students{[}\textquotesingle{}Amadeus\textquotesingle{}{]}{[}\textquotesingle{}History\textquotesingle{}{]}}
  to retrive Amadeus score on the history test.
\end{enumerate}

{[}HINT: The values in a dictionary can be dictionaries.{]}

    \begin{Verbatim}[commandchars=\\\{\}]
{\color{incolor}In [{\color{incolor}38}]:} \PY{n}{students}\PY{o}{=}\PY{p}{\PYZob{}}\PY{l+s+s2}{\PYZdq{}}\PY{l+s+s2}{Amadeus}\PY{l+s+s2}{\PYZdq{}}\PY{p}{:}\PY{n}{amadeus}\PY{p}{,}\PY{l+s+s2}{\PYZdq{}}\PY{l+s+s2}{Rosa}\PY{l+s+s2}{\PYZdq{}}\PY{p}{:}\PY{n}{rosa}\PY{p}{,}\PY{l+s+s2}{\PYZdq{}}\PY{l+s+s2}{Mona}\PY{l+s+s2}{\PYZdq{}}\PY{p}{:}\PY{n}{mona}\PY{p}{,}\PY{l+s+s2}{\PYZdq{}}\PY{l+s+s2}{Ludwig}\PY{l+s+s2}{\PYZdq{}}\PY{p}{:}\PY{n}{ludwig}\PY{p}{\PYZcb{}}
\end{Verbatim}


    \begin{Verbatim}[commandchars=\\\{\}]
{\color{incolor}In [{\color{incolor}39}]:} \PY{n}{students}
\end{Verbatim}


\begin{Verbatim}[commandchars=\\\{\}]
{\color{outcolor}Out[{\color{outcolor}39}]:} \{'Amadeus': \{'Name': 'Amadeus', 'Gender': 'Male', 'Algebra': 8, 'History': 13\},
          'Rosa': \{'Name': 'Rosa', 'Gender': 'Female', 'Algebra': 19, 'History': 22\},
          'Mona': \{'Name': 'Mona', 'Gender': 'Female', 'Algebra': 6, 'History': 27\},
          'Ludwig': \{'Name': 'Ludwig', 'Gender': 'Other', 'Algebra': 12, 'History': 18\}\}
\end{Verbatim}
            
    \begin{enumerate}
\def\labelenumi{\alph{enumi})}
\setcounter{enumi}{3}
\tightlist
\item
  Add the new male student Karl to the dictionary \texttt{students}.
  Karl scored 14 on the Algebra exam and 10 on the History exam.
\end{enumerate}

    \begin{Verbatim}[commandchars=\\\{\}]
{\color{incolor}In [{\color{incolor}40}]:} \PY{n}{karl}\PY{o}{=}\PY{p}{\PYZob{}}\PY{l+s+s2}{\PYZdq{}}\PY{l+s+s2}{Name}\PY{l+s+s2}{\PYZdq{}}\PY{p}{:}\PY{l+s+s2}{\PYZdq{}}\PY{l+s+s2}{Karl}\PY{l+s+s2}{\PYZdq{}}\PY{p}{,}\PY{l+s+s2}{\PYZdq{}}\PY{l+s+s2}{Gender}\PY{l+s+s2}{\PYZdq{}}\PY{p}{:}\PY{l+s+s2}{\PYZdq{}}\PY{l+s+s2}{Male}\PY{l+s+s2}{\PYZdq{}}\PY{p}{,}\PY{l+s+s2}{\PYZdq{}}\PY{l+s+s2}{Algebra}\PY{l+s+s2}{\PYZdq{}}\PY{p}{:}\PY{l+m+mi}{14}\PY{p}{,}\PY{l+s+s2}{\PYZdq{}}\PY{l+s+s2}{History}\PY{l+s+s2}{\PYZdq{}}\PY{p}{:}\PY{l+m+mi}{10}\PY{p}{\PYZcb{}}
         \PY{n}{students}\PY{p}{[}\PY{l+s+s2}{\PYZdq{}}\PY{l+s+s2}{Karl}\PY{l+s+s2}{\PYZdq{}}\PY{p}{]}\PY{o}{=}\PY{n}{karl}
         \PY{n}{students}
\end{Verbatim}


\begin{Verbatim}[commandchars=\\\{\}]
{\color{outcolor}Out[{\color{outcolor}40}]:} \{'Amadeus': \{'Name': 'Amadeus', 'Gender': 'Male', 'Algebra': 8, 'History': 13\},
          'Rosa': \{'Name': 'Rosa', 'Gender': 'Female', 'Algebra': 19, 'History': 22\},
          'Mona': \{'Name': 'Mona', 'Gender': 'Female', 'Algebra': 6, 'History': 27\},
          'Ludwig': \{'Name': 'Ludwig', 'Gender': 'Other', 'Algebra': 12, 'History': 18\},
          'Karl': \{'Name': 'Karl', 'Gender': 'Male', 'Algebra': 14, 'History': 10\}\}
\end{Verbatim}
            
    \begin{enumerate}
\def\labelenumi{\alph{enumi})}
\setcounter{enumi}{4}
\tightlist
\item
  Use a \texttt{for}-loop to print out the names and scores of all
  students on the screen. The output should look like something this
  (the order of the students doesn't matter):
\end{enumerate}

\begin{quote}
\texttt{Student\ Amadeus\ scored\ 8\ on\ the\ Algebra\ exam\ and\ 13\ on\ the\ History\ exam}
\texttt{Student\ Rosa\ scored\ 19\ on\ the\ Algebra\ exam\ and\ 22\ on\ the\ History\ exam}
...
\end{quote}

{[}Hint: Dictionaries are iterables, also, check out the \texttt{items}
function for dictionaries.{]}

    \begin{Verbatim}[commandchars=\\\{\}]
{\color{incolor}In [{\color{incolor}41}]:} \PY{k}{for} \PY{n}{k}\PY{p}{,} \PY{n}{v} \PY{o+ow}{in} \PY{n}{students}\PY{o}{.}\PY{n}{items}\PY{p}{(}\PY{p}{)}\PY{p}{:}
             \PY{c+c1}{\PYZsh{} Your code goes here.}
             \PY{n+nb}{print}\PY{p}{(}\PY{l+s+s2}{\PYZdq{}}\PY{l+s+s2}{Student }\PY{l+s+si}{\PYZob{}0\PYZcb{}}\PY{l+s+s2}{ scored }\PY{l+s+si}{\PYZob{}1\PYZcb{}}\PY{l+s+s2}{ on the Algebra exam and }\PY{l+s+si}{\PYZob{}2\PYZcb{}}\PY{l+s+s2}{ on the History exam}\PY{l+s+s2}{\PYZdq{}}
                   \PY{o}{.}\PY{n}{format}\PY{p}{(}\PY{n}{students}\PY{p}{[}\PY{n}{k}\PY{p}{]}\PY{p}{[}\PY{l+s+s2}{\PYZdq{}}\PY{l+s+s2}{Name}\PY{l+s+s2}{\PYZdq{}}\PY{p}{]}\PY{p}{,}\PY{n}{students}\PY{p}{[}\PY{n}{k}\PY{p}{]}\PY{p}{[}\PY{l+s+s2}{\PYZdq{}}\PY{l+s+s2}{Algebra}\PY{l+s+s2}{\PYZdq{}}\PY{p}{]}\PY{p}{,}\PY{n}{students}\PY{p}{[}\PY{n}{k}\PY{p}{]}\PY{p}{[}\PY{l+s+s2}{\PYZdq{}}\PY{l+s+s2}{History}\PY{l+s+s2}{\PYZdq{}}\PY{p}{]}\PY{p}{)}\PY{p}{)}
             
\end{Verbatim}


    \begin{Verbatim}[commandchars=\\\{\}]
Student Amadeus scored 8 on the Algebra exam and 13 on the History exam
Student Rosa scored 19 on the Algebra exam and 22 on the History exam
Student Mona scored 6 on the Algebra exam and 27 on the History exam
Student Ludwig scored 12 on the Algebra exam and 18 on the History exam
Student Karl scored 14 on the Algebra exam and 10 on the History exam

    \end{Verbatim}

    \begin{enumerate}
\def\labelenumi{\alph{enumi})}
\setcounter{enumi}{5}
\tightlist
\item
  Use a dict comprehension and the lists \texttt{names} and
  \texttt{short\_long} from assignment 2 to create a dictionary of names
  and wether they are short or long. The result should be a dictionary
  equivalent to \{'Forex':'long', 'Tesco':'long', ...\}.
\end{enumerate}

{[}Note: Remember that dictionaries in Python are unordered and that the
order of the pairs in the above dictionary is arbitrary, you might not
get the same order, this is fine.{]}

    \begin{Verbatim}[commandchars=\\\{\}]
{\color{incolor}In [{\color{incolor}42}]:} \PY{p}{\PYZob{}}\PY{n}{key}\PY{p}{:}\PY{n}{value} \PY{k}{for} \PY{p}{(}\PY{n}{key}\PY{p}{,}\PY{n}{value}\PY{p}{)} \PY{o+ow}{in} \PY{n+nb}{zip}\PY{p}{(}\PY{n}{names}\PY{p}{,}\PY{n}{short\PYZus{}long}\PY{p}{)}\PY{p}{\PYZcb{}}
\end{Verbatim}


\begin{Verbatim}[commandchars=\\\{\}]
{\color{outcolor}Out[{\color{outcolor}42}]:} \{'Tesco': 'long', 'Forex': 'long', 'Alonzo': 'long', 'Zeno': 'short'\}
\end{Verbatim}
            
    \subsubsection{5. Introductory file I/O}\label{introductory-file-io}

File I/O in Python is a bit more general than what most R programmers
are used to. In R, reading and writing files are usually performed using
file type specific functions such as \texttt{read.csv} while in Python
we usually start with reading standard text files. However, there are
lots of specialized functions for different file types in Python as
well, especially when using the
\textbf{\href{http://pandas.pydata.org/}{pandas}} library which is built
around a datatype similar to R DataFrames. Pandas will not be covered in
this course though.

{[}Literature: Files are introduced in LP part II chapter 4 and chapter
9.{]}

    The file \texttt{students.tsv} contains tab separated values
corresponding to the students in previous assigments.

\begin{enumerate}
\def\labelenumi{\alph{enumi})}
\tightlist
\item
  Iterate over the file, line by line, and print each line. The result
  should be something like this:
\end{enumerate}

\begin{quote}
\texttt{Amadeus\ \ Male\ \ \ \ 8\ \ \ 13}
\texttt{Rosa\ Female\ \ 19\ \ 22} ...
\end{quote}

The file should be closed when reading is complete.

{[}Hint: Files are iterable in Python.{]}

    \begin{Verbatim}[commandchars=\\\{\}]
{\color{incolor}In [{\color{incolor}43}]:} \PY{n}{f}\PY{o}{=}\PY{n+nb}{open}\PY{p}{(}\PY{l+s+s2}{\PYZdq{}}\PY{l+s+s2}{students.tsv}\PY{l+s+s2}{\PYZdq{}}\PY{p}{,}\PY{l+s+s2}{\PYZdq{}}\PY{l+s+s2}{r}\PY{l+s+s2}{\PYZdq{}}\PY{p}{)}
         \PY{n}{read\PYZus{}data}\PY{o}{=}\PY{n}{f}\PY{o}{.}\PY{n}{read}\PY{p}{(}\PY{p}{)}
         \PY{n+nb}{print} \PY{p}{(}\PY{n}{read\PYZus{}data}\PY{p}{)}
         \PY{n}{f}\PY{o}{.}\PY{n}{close}\PY{p}{(}\PY{p}{)}
         \PY{n}{f}\PY{o}{.}\PY{n}{closed}
\end{Verbatim}


    \begin{Verbatim}[commandchars=\\\{\}]
Amadeus	Male	8	13
Rosa	Female	19	22
Mona	Female	6	27
Ludwig	Other	12	18
Karl	Male	14	10

    \end{Verbatim}

\begin{Verbatim}[commandchars=\\\{\}]
{\color{outcolor}Out[{\color{outcolor}43}]:} True
\end{Verbatim}
            
    \begin{enumerate}
\def\labelenumi{\alph{enumi})}
\setcounter{enumi}{1}
\tightlist
\item
  Working with many files can be problematic, especially when you forget
  to close files or errors interrupt programs before files are closed.
  Python thus has a special \texttt{with} statement which automatically
  closes files for you, even if an error occurs. Redo the assignment
  above using the \texttt{with} statement.
\end{enumerate}

{[}Literature: With is introduced in LP part II chapter 9 page 294.{]}

    \begin{Verbatim}[commandchars=\\\{\}]
{\color{incolor}In [{\color{incolor}44}]:} \PY{k}{with} \PY{n+nb}{open}\PY{p}{(}\PY{l+s+s2}{\PYZdq{}}\PY{l+s+s2}{students.tsv}\PY{l+s+s2}{\PYZdq{}}\PY{p}{,}\PY{l+s+s2}{\PYZdq{}}\PY{l+s+s2}{r}\PY{l+s+s2}{\PYZdq{}}\PY{p}{)} \PY{k}{as} \PY{n}{f}\PY{p}{:}
             \PY{n}{read\PYZus{}data}\PY{o}{=}\PY{n}{f}\PY{o}{.}\PY{n}{read}\PY{p}{(}\PY{p}{)}
         \PY{n}{f}\PY{o}{.}\PY{n}{closed}
         \PY{n+nb}{print}\PY{p}{(}\PY{n}{read\PYZus{}data}\PY{p}{)}
\end{Verbatim}


    \begin{Verbatim}[commandchars=\\\{\}]
Amadeus	Male	8	13
Rosa	Female	19	22
Mona	Female	6	27
Ludwig	Other	12	18
Karl	Male	14	10

    \end{Verbatim}

    \begin{enumerate}
\def\labelenumi{\alph{enumi})}
\setcounter{enumi}{2}
\tightlist
\item
  Recreate the dictionary from assignment the previous assignment by
  reading the data from the file. Using a dedicated csv-reader is not
  permitted.
\end{enumerate}

    \begin{Verbatim}[commandchars=\\\{\}]
{\color{incolor}In [{\color{incolor}45}]:} \PY{n}{s}\PY{o}{=}\PY{p}{\PYZob{}}\PY{p}{\PYZcb{}}
         \PY{k}{with} \PY{n+nb}{open}\PY{p}{(}\PY{l+s+s2}{\PYZdq{}}\PY{l+s+s2}{students.tsv}\PY{l+s+s2}{\PYZdq{}}\PY{p}{,}\PY{l+s+s2}{\PYZdq{}}\PY{l+s+s2}{r}\PY{l+s+s2}{\PYZdq{}}\PY{p}{)} \PY{k}{as} \PY{n}{f}\PY{p}{:}
             \PY{k}{for} \PY{n}{line} \PY{o+ow}{in} \PY{n}{f}\PY{p}{:}
                 \PY{n}{s}\PY{p}{[}\PY{n}{line}\PY{o}{.}\PY{n}{split}\PY{p}{(}\PY{p}{)}\PY{p}{[}\PY{l+m+mi}{0}\PY{p}{]}\PY{p}{]}\PY{o}{=}\PY{p}{\PYZob{}}
                 \PY{l+s+s2}{\PYZdq{}}\PY{l+s+s2}{Name}\PY{l+s+s2}{\PYZdq{}}\PY{p}{:}\PY{n}{line}\PY{o}{.}\PY{n}{split}\PY{p}{(}\PY{p}{)}\PY{p}{[}\PY{l+m+mi}{0}\PY{p}{]}\PY{p}{,}
                 \PY{l+s+s2}{\PYZdq{}}\PY{l+s+s2}{Gender}\PY{l+s+s2}{\PYZdq{}}\PY{p}{:}\PY{n}{line}\PY{o}{.}\PY{n}{split}\PY{p}{(}\PY{p}{)}\PY{p}{[}\PY{l+m+mi}{1}\PY{p}{]}\PY{p}{,}
                 \PY{l+s+s2}{\PYZdq{}}\PY{l+s+s2}{Algebra}\PY{l+s+s2}{\PYZdq{}}\PY{p}{:}\PY{n}{line}\PY{o}{.}\PY{n}{split}\PY{p}{(}\PY{p}{)}\PY{p}{[}\PY{l+m+mi}{2}\PY{p}{]}\PY{p}{,}
                 \PY{l+s+s2}{\PYZdq{}}\PY{l+s+s2}{History}\PY{l+s+s2}{\PYZdq{}}\PY{p}{:}\PY{n}{line}\PY{o}{.}\PY{n}{split}\PY{p}{(}\PY{p}{)}\PY{p}{[}\PY{l+m+mi}{3}\PY{p}{]}\PY{p}{\PYZcb{}}
                 
         \PY{k}{for} \PY{n}{k}\PY{p}{,}\PY{n}{v} \PY{o+ow}{in} \PY{n}{s}\PY{o}{.}\PY{n}{items}\PY{p}{(}\PY{p}{)}\PY{p}{:}
             \PY{n+nb}{print}\PY{p}{(}\PY{n}{k}\PY{p}{,}\PY{n}{v}\PY{p}{)}
\end{Verbatim}


    \begin{Verbatim}[commandchars=\\\{\}]
Amadeus \{'Name': 'Amadeus', 'Gender': 'Male', 'Algebra': '8', 'History': '13'\}
Rosa \{'Name': 'Rosa', 'Gender': 'Female', 'Algebra': '19', 'History': '22'\}
Mona \{'Name': 'Mona', 'Gender': 'Female', 'Algebra': '6', 'History': '27'\}
Ludwig \{'Name': 'Ludwig', 'Gender': 'Other', 'Algebra': '12', 'History': '18'\}
Karl \{'Name': 'Karl', 'Gender': 'Male', 'Algebra': '14', 'History': '10'\}

    \end{Verbatim}

    \begin{enumerate}
\def\labelenumi{\alph{enumi})}
\setcounter{enumi}{3}
\tightlist
\item
  Using the dictionary above, write sentences from task 4e above to a
  new file, called \texttt{students.txt}.
\end{enumerate}

    \begin{Verbatim}[commandchars=\\\{\}]
{\color{incolor}In [{\color{incolor}46}]:} \PY{k}{with} \PY{n+nb}{open}\PY{p}{(}\PY{l+s+s1}{\PYZsq{}}\PY{l+s+s1}{students.txt}\PY{l+s+s1}{\PYZsq{}}\PY{p}{,} \PY{l+s+s1}{\PYZsq{}}\PY{l+s+s1}{a+}\PY{l+s+s1}{\PYZsq{}}\PY{p}{)} \PY{k}{as} \PY{n}{the\PYZus{}file}\PY{p}{:}
             \PY{k}{for} \PY{n}{k}\PY{p}{,}\PY{n}{v} \PY{o+ow}{in} \PY{n}{s}\PY{o}{.}\PY{n}{items}\PY{p}{(}\PY{p}{)}\PY{p}{:}
                 \PY{n}{the\PYZus{}file}\PY{o}{.}\PY{n}{writelines}\PY{p}{(}\PY{l+s+s2}{\PYZdq{}}\PY{l+s+s2}{Student }\PY{l+s+si}{\PYZob{}0\PYZcb{}}\PY{l+s+s2}{ scored }\PY{l+s+si}{\PYZob{}1\PYZcb{}}\PY{l+s+s2}{ on the Algebra exam and }\PY{l+s+si}{\PYZob{}2\PYZcb{}}\PY{l+s+s2}{ on the History exam}\PY{l+s+s2}{\PYZdq{}}
                   \PY{o}{.}\PY{n}{format}\PY{p}{(}\PY{n}{s}\PY{p}{[}\PY{n}{k}\PY{p}{]}\PY{p}{[}\PY{l+s+s2}{\PYZdq{}}\PY{l+s+s2}{Name}\PY{l+s+s2}{\PYZdq{}}\PY{p}{]}\PY{p}{,}\PY{n}{s}\PY{p}{[}\PY{n}{k}\PY{p}{]}\PY{p}{[}\PY{l+s+s2}{\PYZdq{}}\PY{l+s+s2}{Algebra}\PY{l+s+s2}{\PYZdq{}}\PY{p}{]}\PY{p}{,}\PY{n}{s}\PY{p}{[}\PY{n}{k}\PY{p}{]}\PY{p}{[}\PY{l+s+s2}{\PYZdq{}}\PY{l+s+s2}{History}\PY{l+s+s2}{\PYZdq{}}\PY{p}{]}\PY{p}{)}\PY{p}{)}
                     
                 \PY{n}{the\PYZus{}file}\PY{o}{.}\PY{n}{writelines}\PY{p}{(}\PY{l+s+s2}{\PYZdq{}}\PY{l+s+se}{\PYZbs{}n}\PY{l+s+s2}{\PYZdq{}}\PY{p}{)}
\end{Verbatim}



    % Add a bibliography block to the postdoc
    
    
    
    \end{document}
